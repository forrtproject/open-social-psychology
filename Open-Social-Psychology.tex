% Options for packages loaded elsewhere
\PassOptionsToPackage{unicode}{hyperref}
\PassOptionsToPackage{hyphens}{url}
\PassOptionsToPackage{dvipsnames,svgnames,x11names}{xcolor}
%
\documentclass[
  letterpaper,
]{book}

\usepackage{amsmath,amssymb}
\usepackage{iftex}
\ifPDFTeX
  \usepackage[T1]{fontenc}
  \usepackage[utf8]{inputenc}
  \usepackage{textcomp} % provide euro and other symbols
\else % if luatex or xetex
  \usepackage{unicode-math}
  \defaultfontfeatures{Scale=MatchLowercase}
  \defaultfontfeatures[\rmfamily]{Ligatures=TeX,Scale=1}
\fi
\usepackage{lmodern}
\ifPDFTeX\else  
    % xetex/luatex font selection
\fi
% Use upquote if available, for straight quotes in verbatim environments
\IfFileExists{upquote.sty}{\usepackage{upquote}}{}
\IfFileExists{microtype.sty}{% use microtype if available
  \usepackage[]{microtype}
  \UseMicrotypeSet[protrusion]{basicmath} % disable protrusion for tt fonts
}{}
\makeatletter
\@ifundefined{KOMAClassName}{% if non-KOMA class
  \IfFileExists{parskip.sty}{%
    \usepackage{parskip}
  }{% else
    \setlength{\parindent}{0pt}
    \setlength{\parskip}{6pt plus 2pt minus 1pt}}
}{% if KOMA class
  \KOMAoptions{parskip=half}}
\makeatother
\usepackage{xcolor}
\setlength{\emergencystretch}{3em} % prevent overfull lines
\setcounter{secnumdepth}{5}
% Make \paragraph and \subparagraph free-standing
\ifx\paragraph\undefined\else
  \let\oldparagraph\paragraph
  \renewcommand{\paragraph}[1]{\oldparagraph{#1}\mbox{}}
\fi
\ifx\subparagraph\undefined\else
  \let\oldsubparagraph\subparagraph
  \renewcommand{\subparagraph}[1]{\oldsubparagraph{#1}\mbox{}}
\fi


\providecommand{\tightlist}{%
  \setlength{\itemsep}{0pt}\setlength{\parskip}{0pt}}\usepackage{longtable,booktabs,array}
\usepackage{calc} % for calculating minipage widths
% Correct order of tables after \paragraph or \subparagraph
\usepackage{etoolbox}
\makeatletter
\patchcmd\longtable{\par}{\if@noskipsec\mbox{}\fi\par}{}{}
\makeatother
% Allow footnotes in longtable head/foot
\IfFileExists{footnotehyper.sty}{\usepackage{footnotehyper}}{\usepackage{footnote}}
\makesavenoteenv{longtable}
\usepackage{graphicx}
\makeatletter
\def\maxwidth{\ifdim\Gin@nat@width>\linewidth\linewidth\else\Gin@nat@width\fi}
\def\maxheight{\ifdim\Gin@nat@height>\textheight\textheight\else\Gin@nat@height\fi}
\makeatother
% Scale images if necessary, so that they will not overflow the page
% margins by default, and it is still possible to overwrite the defaults
% using explicit options in \includegraphics[width, height, ...]{}
\setkeys{Gin}{width=\maxwidth,height=\maxheight,keepaspectratio}
% Set default figure placement to htbp
\makeatletter
\def\fps@figure{htbp}
\makeatother
% definitions for citeproc citations
\NewDocumentCommand\citeproctext{}{}
\NewDocumentCommand\citeproc{mm}{%
  \begingroup\def\citeproctext{#2}\cite{#1}\endgroup}
\makeatletter
 % allow citations to break across lines
 \let\@cite@ofmt\@firstofone
 % avoid brackets around text for \cite:
 \def\@biblabel#1{}
 \def\@cite#1#2{{#1\if@tempswa , #2\fi}}
\makeatother
\newlength{\cslhangindent}
\setlength{\cslhangindent}{1.5em}
\newlength{\csllabelwidth}
\setlength{\csllabelwidth}{3em}
\newenvironment{CSLReferences}[2] % #1 hanging-indent, #2 entry-spacing
 {\begin{list}{}{%
  \setlength{\itemindent}{0pt}
  \setlength{\leftmargin}{0pt}
  \setlength{\parsep}{0pt}
  % turn on hanging indent if param 1 is 1
  \ifodd #1
   \setlength{\leftmargin}{\cslhangindent}
   \setlength{\itemindent}{-1\cslhangindent}
  \fi
  % set entry spacing
  \setlength{\itemsep}{#2\baselineskip}}}
 {\end{list}}
\usepackage{calc}
\newcommand{\CSLBlock}[1]{\hfill\break\parbox[t]{\linewidth}{\strut\ignorespaces#1\strut}}
\newcommand{\CSLLeftMargin}[1]{\parbox[t]{\csllabelwidth}{\strut#1\strut}}
\newcommand{\CSLRightInline}[1]{\parbox[t]{\linewidth - \csllabelwidth}{\strut#1\strut}}
\newcommand{\CSLIndent}[1]{\hspace{\cslhangindent}#1}

\usepackage{fancyhdr}
\usepackage{graphicx}
\usepackage{eso-pic}
\usepackage{tikz}
\AtBeginDocument{\thispagestyle{empty}\begin{tikzpicture}[remember picture,overlay] \node at (current page.center) [yshift=1cm] {\includegraphics[width=0.75\paperwidth,height=0.9\paperheight,keepaspectratio]{resources/cover.png}}; \end{tikzpicture}\clearpage}
\makeatletter
\@ifpackageloaded{bookmark}{}{\usepackage{bookmark}}
\makeatother
\makeatletter
\@ifpackageloaded{caption}{}{\usepackage{caption}}
\AtBeginDocument{%
\ifdefined\contentsname
  \renewcommand*\contentsname{Table of contents}
\else
  \newcommand\contentsname{Table of contents}
\fi
\ifdefined\listfigurename
  \renewcommand*\listfigurename{List of Figures}
\else
  \newcommand\listfigurename{List of Figures}
\fi
\ifdefined\listtablename
  \renewcommand*\listtablename{List of Tables}
\else
  \newcommand\listtablename{List of Tables}
\fi
\ifdefined\figurename
  \renewcommand*\figurename{Figure}
\else
  \newcommand\figurename{Figure}
\fi
\ifdefined\tablename
  \renewcommand*\tablename{Table}
\else
  \newcommand\tablename{Table}
\fi
}
\@ifpackageloaded{float}{}{\usepackage{float}}
\floatstyle{ruled}
\@ifundefined{c@chapter}{\newfloat{codelisting}{h}{lop}}{\newfloat{codelisting}{h}{lop}[chapter]}
\floatname{codelisting}{Listing}
\newcommand*\listoflistings{\listof{codelisting}{List of Listings}}
\makeatother
\makeatletter
\makeatother
\makeatletter
\@ifpackageloaded{caption}{}{\usepackage{caption}}
\@ifpackageloaded{subcaption}{}{\usepackage{subcaption}}
\makeatother
\ifLuaTeX
  \usepackage{selnolig}  % disable illegal ligatures
\fi
\usepackage{bookmark}

\IfFileExists{xurl.sty}{\usepackage{xurl}}{} % add URL line breaks if available
\urlstyle{same} % disable monospaced font for URLs
\hypersetup{
  pdftitle={Open Social Psychology},
  pdfauthor={Rima-Maria Rahal},
  colorlinks=true,
  linkcolor={green},
  filecolor={Maroon},
  citecolor={green},
  urlcolor={Blue},
  pdfcreator={LaTeX via pandoc}}

\title{Open Social Psychology}
\author{Rima-Maria Rahal}
\date{2025-08-27}

\begin{document}
\frontmatter
\maketitle

\renewcommand*\contentsname{Table of contents}
{
\hypersetup{linkcolor=}
\setcounter{tocdepth}{2}
\tableofcontents
}
\mainmatter
\bookmarksetup{startatroot}

\chapter*{\texorpdfstring{{Preface}}{Preface}}\label{preface}
\addcontentsline{toc}{chapter}{{Preface}}

\markboth{{Preface}}{{Preface}}

Social psychology is built on a strong set of classical research
paradigms and findings, featured in many of the textbooks, syllabi,
online courses and teaching guides that aspiring psychologists study
with and established psychologists use as teaching resources. However,
the common body of knowledge that social psychology relies on is
undergoing change. Modern research methods and changing attitudes
towards permissible research practices bring about social psychological
research that looks different today than it used. This book is dedicated
to tracing some of these changes, and to offering a version of record of
the changing perceptions and interpretations of classic social
psychology in the light of it's contemporary counterpart. As such, this
study book is a snapshot of how we see social psychology today.

Because it tends to be difficult to keep teaching and study materials up
to date with emerging trends and debates, we see this study book as an
addition to traditional educational resources in social psychology. It
is published as an Open Educational Resource to aid the accessibility of
this knowledge for all, and to be adapted to teachers' and learners'
needs as they dive into what social psychology has to offer.

\bookmarksetup{startatroot}

\chapter*{\texorpdfstring{{How this Book Came to
Be}}{How this Book Came to Be}}\label{how-this-book-came-to-be}
\addcontentsline{toc}{chapter}{{How this Book Came to Be}}

\markboth{{How this Book Came to Be}}{{How this Book Came to Be}}

{written by Flávio Azevedo and Rima-Maria Rahal}

Social psychology is devoted to studying how individuals behave, think
and feel within their social contexts. The field is therefore, by its
very nature, set up for collaborative work. Leveraging the social
context in which knowledge is generated is built in to the assumptions
and interests that social psychology pursues. This fundamental attitude
towards social embeddedness of knowledge is mirrored in the process by
which this study book came to be.

It started by bringing together the work of students at Heidelberg
University during the winter term of 2023. In the scope of classwork,
they engaged with classical findings of social psychology, and discussed
recent attempts to reengage with these classics. These works are the
basis of the current book.

Researchers working on (areas related to) social psychology then revised
these chapters. Through engaging the communities at the Big Team Science
Conference 2024
(\href{https://bigteamscienceconference.github.io}{BTScon}), the 2025
annual meeting of the Society for the Improvement of Psychological
Science (\href{http://improvingpsych.org}{SIPS}) and the Framework for
Open and Reproducible Research Training
(\href{https://forrt.org}{FORRT}), we found collaborators willing to
contribute their knowledge and expertise to turning chapter drafts into
an approachable and fact checked resource.

\includegraphics{FORRT.png}

The creation of this book was also supported by the German
Reproducibility Network (\href{https://reproducibilitynetwork.de}{GRN}).

By co-creating educational content with diverse participants rather than
relying solely on traditional authority figures, the process of writing
this book explicitly built in diverse perspectives and lived experiences
of groups who may otherwise not have access to such contribution
opportunities. This process promoted cross-cultural scholarly exchange
on replication and reproducibility in social psychology, and made this
educational resource more inclusive by reflecting diverse perspectives.

In sum, this volume offers diverse perspectives on a shared target
topic: Changing perceptions of classical social psychological research.

\bookmarksetup{startatroot}

\chapter*{\texorpdfstring{{How to Use this
Book}}{How to Use this Book}}\label{how-to-use-this-book}
\addcontentsline{toc}{chapter}{{How to Use this Book}}

\markboth{{How to Use this Book}}{{How to Use this Book}}

{written by Melissa Engelbart and Rima-Maria Rahal}

This book contains several types of resources: narrative text,
definitions and questions for reflection, as well as references.

In fifteen chapters, we provide narrative summaries about classical
research in social psychology and its modern follow-up. Often, this
means we include new attempts to show the same finding (replication
attempts) or meta-analytical work that brings together a lot of evidence
from different sources regarding a certain hypothesis. Each chapter
contains an overview of the classic study, a summary of important work
thereafter, as well as a discussion of the evidence, experiments or
analyses conducted. We then attempt to draw conclusions about the tested
hypotheses.

Because this volume is targeted at students, we provide definitions of
key terms, preceded by {\#definition} and displayed like this:

\begin{quote}
{\#definition} Replication

An attempt to find the same result as a previous study in a new data
set.
\end{quote}

\begin{quote}
{\#definition} Meta-Analysis

An analysis that brings together evidence from several individual
studies or experiments to estimate an overall effect across the
available evidence.
\end{quote}

We have aimed at providing a critical but neutral perspective to the
classical and modern studies of social psychology discussed in the texts
of this volume. To help you develop your own perspective and a
well-reflected attitude towards this work, you will find guiding
questions and suggestions that might prompt you to think more deeply
about what you read throughout the book. The guiding questions cover
topics such as the research and publication process itself and it's
influence on research, the interpretation of data in general, as well as
the experimental operationalization of theoretical questions. Moreover,
to help you consider potential applications of the findings and theories
discussed, these questions sometimes ask you to think of examples or
consequences in real life.

You'll recognize these prompts by the preceding {\#yourturn.} Here is an
example of what these questions look like:

\begin{quote}
{\#yourturn}

Do you think you might find such questions for reflection useful?
\end{quote}

Finally, we have enabled the option to collaboratively annotate this
work using \href{https://web.hypothes.is/}{hypothesis} (note that this
is how links are formatted in this book) in the online version. Your
annotations will be visible to others, and others will be able to see
yours, so that we can build a better learning experience using this book
together.

To read up on the original research we cite in this book, such as from
Vazire (\citeproc{ref-vazire2018}{2018}), you can hover over or click on
the references provided.

Feel free to make use of the resources in this book as you see fit. Our
hope is that they will support you in building a well-reflected opinion
about the existing body of knowledge in social psychology.

\bookmarksetup{startatroot}

\chapter*{\texorpdfstring{{Introduction}}{Introduction}}\label{introduction}
\addcontentsline{toc}{chapter}{{Introduction}}

\markboth{{Introduction}}{{Introduction}}

{written by Rima-Maria Rahal}

\section*{The Role of Change for Scientific
Discovery}\label{the-role-of-change-for-scientific-discovery}
\addcontentsline{toc}{section}{The Role of Change for Scientific
Discovery}

\markright{The Role of Change for Scientific Discovery}

Much of science capitalizes on change. It is the engine that drives
progress and the expansion of knowledge (see
\citeproc{ref-kuhn1962structure}{Kuhn 1962};
\citeproc{ref-popper1959logic}{Popper 1959}). Embracing change means
taking established theories and challenging them to explore new
directions. Changing perspectives, questioning the status quo, refining
existing concepts, and adapting to new evidence provide the stuff that
makes breakthroughs or new insights. In essence, change in science
represents taking steps forward, toward greater insight and reality
checks for the challenges we face. In other words, to push the
boundaries of what we know, we must make change.

\begin{quote}
{\#yourturn}

What instance of change regarding science have you recently heard about?
Consider reports of breakthroughs you might have seen in the news or
stories you saw on social media.
\end{quote}

In the past decade, Open Science has made change, by transforming
research practices to promote transparency, reproducibility, and
collaboration in scientific endeavors. By fostering a culture of
openness and collaboration, Open Science has brought about a paradigm
shift in research methodologies, paving the way for more robust and
reliable scientific discoveries
(\citeproc{ref-munafo2017manifesto}{Munafò et al. 2017};
\citeproc{ref-vazire2022credibility}{Vazire, Schiavone, and Bottesini
2022}). It is certainly no small feat to fundamentally reform how
research is done, and yet we have seen significant change towards Open
practices (\citeproc{ref-kidwell2016badges}{Kidwell et al. 2016};
\citeproc{ref-chambers2019registered}{Chambers 2019};
\citeproc{ref-christensen2020open}{Christensen et al. 2020}).

\begin{quote}
{\#definition} Open Science

An overhead term for a number of practices to make research more
transparent, such as making the data a research is project is based on
available to others.
\end{quote}

\section*{Challenges of Making
Change}\label{challenges-of-making-change}
\addcontentsline{toc}{section}{Challenges of Making Change}

\markright{Challenges of Making Change}

Change can be a challenge because it disrupts established norms, habits,
and power structures. This often means that individuals and groups might
be hesitant to embrace change. Open Science as a reform to refocus on
good research practice had to work with this difficulty of making
change, where new methods, theories, or technologies often encounter
skepticism and opposition from the scientific community. Open Science
promotes transparency, data sharing, and collaborative research, which
can expose flaws underlying previously held beliefs or reveal
alternative interpretations. This shift can create debates about
long-held ideas and established practices, which are scrutinized and
potentially overturned. Established researchers may be reluctant to
abandon familiar paradigms, and institutions may resist reallocating
resources or altering well-known processes. Sometimes, inertia of
traditional practices and fear of uncertainty can slow the adoption of
innovative approaches, despite their potential to advance knowledge and
solve pressing problems.

\begin{quote}
{\#yourturn}

Consider a big change you have experienced. Was it easy to adapt to this
change?
\end{quote}

However, a questioning attitude and focus on methodological rigor and
good practice also enhance the robustness and reliability of scientific
conclusions by fostering an environment where continuous re-evaluation
is encouraged. Thus, Open Science exemplifies how embracing change can
lead to a more dynamic and resilient understanding of the world, even as
it unsettles the familiar foundations of scientific consensus.

Change often implies the potential for a changed perception of what used
to be, particularly in comparison to what is now. This is also the case
in the scope of changes assosciated with Open Science. In particular,
what were once considered unassailable facts can become contested or
uncertain as new methodologies, data, and technologies challenge
established knowledge. This is where our focus lies in this book:
reporting on classical studies in social psychology and the change in
how they are seen now, following a wave of additional research (often
with an Open Science flavor).

\begin{quote}
{\#yourturn}

``I was today years old when I found out \ldots{}'' What was the last
long-held belief you had to give up?
\end{quote}

In this spirit, when reading about the changes in perspective about
classics in social psychology, there are two things to embrace:

On the one hand, revisiting classic social psychology studies is a
demonstration of the profound impact they had on the field. Were they
less important and less impactful, these studies would not draw
continued debate, research interest and investment of resources.
Therefore, reading classic studies can give readers a sense of what
matters to social psychological research, from hot topics to hot
paradigms and research methods.

On the other hand, following the course of the academic debate about
these claims, insights and phenomena allows us to hone our skills in
accumulating insights and adjusting our perception of the currently held
beliefs in this area of research. Put differently, tracing efforts to
replicate, to conduct meta-analyses or to establish boundary conditions
to the findings postulated in a certain study mostly reflects
well-intentioned interest in assessing the validity of the claims of the
original study, attempting to produce clarity about our collective
knowledge about the phenomenon of interest. Reassessing classical
studies might require change in opinions, calibration and reflection,
but it can surely spark renewed trust in research and in its ability to
refine and build our joint knowledge.

\bookmarksetup{startatroot}

\chapter{\texorpdfstring{{Pygmalion
Effect}}{Pygmalion Effect}}\label{pygmalion-effect}

{written by Maja Düsenberg (original draft), and Jana Berkessel
(revision)}

\section{The Classic}\label{the-classic}

The Pygmalion Effect is a social psychological phenomenon that
highlights how expectations can influence performance. It was first
demonstrated by Robert Rosenthal and Lenore Jacobson in their seminal
study, \emph{Pygmalion in the Classroom}
(\citeproc{ref-Rosenthal_Jacobson_1968}{Rosenthal and Jacobson 1968}).
The study investigated how teachers' expectations about their students'
potential could shape the students' academic performance.

\begin{quote}
\phantomsection\label{def-pygmalioneffect}{\#definition} Pygmalion
Effect

The phenomenon in which higher expectations from others lead to improved
performance.
\end{quote}

In their experiment, Rosenthal and Jacobson
(\citeproc{ref-Rosenthal_Jacobson_1968}{1968}) told teachers that
certain students were likely to be ``growth spurters'' who were expected
to achieve significant academic improvement over the school year based
on a fabricated test. The students were randomly selected and had no
actual differences in ability compared to their peers.

The results were striking. The so-called ``spurters'' showed
significantly higher gains in IQ scores from the pre-test to the
post-test compared to their control peers. These changes could not be
explained by retesting effects, familiarity with the test, or natural
cognitive development due to aging. Instead, the findings highlighted
the powerful role of teacher expectations in shaping student outcomes.

The study also explored moderating factors. Younger children
demonstrated greater improvements, potentially due to their higher
malleability to external influences. Gender differences were observed,
with girls showing greater increases in reasoning IQ and boys improving
more in verbal IQ, aligning with their respective pretest strengths.
Additionally, while not statistically significant, minority students
appeared to benefit more from positive expectations, with ``more
Mexican-looking'' boys (e.g., darker skin tones) showing particularly
pronounced IQ gains. These results suggest that preconceived notions
based on race and ethnicity may interact with expectation effects.

\begin{quote}
{\#yourturn}

In which other situations could the Pygmalion effect play a role? Think
about situations where your assumptions or expectations about others may
influence their behavior---positively or negatively.
\end{quote}

The implications of the Pygmalion effect extend far beyond the
classroom. In organizational settings, for instance, research has shown
that managers' high expectations for their employees can lead to
improved performance through changes in behavior and increased
self-efficacy (\citeproc{ref-Eden_1990}{Eden 1990}). Similar dynamics
have been observed in therapeutic relationships, where therapists'
beliefs about their clients' potential for progress influence treatment
outcomes (\citeproc{ref-Jenner_1990}{Jenner 1990}), and in healthcare,
where nurses' confidence in their patients' recovery can affect health
results (\citeproc{ref-Learman_et_al_1990}{Learman et al. 1990}). These
examples illustrate how expectations have the power to shape behavior
and performance in diverse domains.

\begin{quote}
{\#yourturn}

Which other social psychological constructs are related to the Pygmalion
effect?
\end{quote}

The Pygmalion effect is closely related to two other psychological
concepts: the self-fulfilling prophecy and self-efficacy. A
self-fulfilling prophecy occurs when an initially false belief or
expectation leads to behaviors that ultimately make the false belief
come true. This concept aligns with the Pygmalion effect, as individuals
may unconsciously alter their actions to align with the expectations
placed upon them. Additionally, self-efficacy, or one's belief in their
ability to succeed in specific situations, plays a key role in mediating
the impact of expectations. When high expectations are communicated,
they can enhance an individual's self-efficacy, reinforcing their
motivation and performance. These interconnected mechanisms help explain
how expectations shape outcomes across various domains.

\section{The Aftermath}\label{the-aftermath}

Thorndike (\citeproc{ref-Thorndike_1968}{1968}) and Snow
(\citeproc{ref-Snow_1969}{1969}) offered early critiques of Rosenthal
and Jacobson's (\citeproc{ref-Rosenthal_Jacobson_1968}{1968}) Pygmalion
study, challenging its methodology, data analysis, and conclusions.
Thorndike (\citeproc{ref-Thorndike_1968}{1968}) focused on issues with
data quality, pointing out inconsistencies such as the implausibly low
IQ scores of some participants, which he described as rendering the
testing meaningless. He argued that the effects of the intervention were
limited primarily to a small group of first- and second-grade students,
raising concerns about the generalizability of the findings. Thorndike
(\citeproc{ref-Thorndike_1968}{1968}) concluded that any observed
effects might have been coincidental rather than genuinely linked to the
intervention.

\begin{quote}
{\#yourturn}

The Pygmalion effect often involves subconscious biases. How do you
think societal stereotypes (e.g., gender, race) might influence the
expectations we hold for others? Can you think of ways to address or
mitigate these biases?
\end{quote}

Snow (\citeproc{ref-Snow_1969}{1969}), similarly skeptical, critiqued
the study's complex experimental design, highlighting incomplete data
and methodological flaws, such as inadequate norms for the youngest
children and those from lower socioeconomic backgrounds. He noted that
20\% of the participants were not retested, an omission unaddressed in
the analysis. Snow (\citeproc{ref-Snow_1969}{1969}) also questioned the
mechanism of teacher influence, pointing out that teachers reportedly
could not recall which students were identified as ``bloomers,''
undermining the study's foundational premise. Snow
(\citeproc{ref-Snow_1969}{1969}) concluded that the study's premature
dissemination in popular media had harmed teachers, parents, and
students by raising unrealistic expectations without robust evidence to
support them.

\begin{quote}
\phantomsection\label{def-specialissue}{\#definition} Special Issue

A collection of articles on a specific topic, typically published
together in a single issue of an academic journal. Special issues are
often edited by guest editors and aim to provide a comprehensive
exploration of the chosen theme or field of study.
\end{quote}

In 2018, the journal \emph{Educational Research and Evaluation}
published a special issue on the Pygmalion effect, just in time for its
50th birthday. In the Editorial, they summarize that despite warranted
criticism of the early studies, research conducted over the past five
decades has refined our understanding of the Pygmalion effect.
Specifically, empirical studies have shown that teachers generally show
a degree of accuracy in their expectations
(\citeproc{ref-Jussim_Harber_2005}{Jussim and Harber 2005}) but tend to
favour students from affluent backgrounds over those from less
privileged ones, while often holding lower expectations for special
needs students (\citeproc{ref-De_Boer_Bosker_Van_der_Werf_2010}{De Boer,
Bosker, and Van der Werf 2010}; \citeproc{ref-Cameron_Cook_2013}{Cameron
and Cook 2013}). Evidence on expectations related to student ethnicity
and gender is more inconsistent, with some studies finding biases---such
as lower expectations for ethnic minority students, boys in reading, and
girls in mathematics---while others do not. Teacher expectations
influence teaching behaviours, such as offering greater opportunities to
learn, asking richer questions, and providing more targeted feedback for
students with higher expectations (\citeproc{ref-Brophy_Good_1970}{J. E.
Brophy and Good 1970}; \citeproc{ref-Rubie-Davies_2007}{C. M.
Rubie-Davies 2007}). These expectations can function as self-fulfilling
prophecies, impacting student outcomes like performance, intelligence,
and motivation. However, the magnitude of these effects varies
significantly across studies (e.g., effect sizes ranging from d = .11,
Raudenbush (\citeproc{ref-Raudenbush_1984}{1984}); to d = .70, Rosenthal
and Rubin (\citeproc{ref-Rosenthal_Rubin_1978}{1978})). Notably,
students who are low achievers, from low-income families, or belong to
ethnic minority groups appear more vulnerable to these effects, and some
teachers are more likely than others to amplify these disparities
(\citeproc{ref-Madon_et_al_1997}{Madon, Jussim, and Eccles 1997};
\citeproc{ref-rubie-davies_teacher_2015}{Christine M. Rubie-Davies et
al. 2015}).

\begin{quote}
\phantomsection\label{def-editorial}{\#definition} Editorial

An introductory article written by the editors of a special issue in an
academic journal. It outlines the purpose, scope, and significance of
the special issue, provides an overview of the included articles, and
often highlights key themes, trends, or gaps in the research field.
\end{quote}

Finally, the Editors underscore the need to view teacher expectations
ecologically, considering the individuality of teachers and students, as
well as the broader contexts of classrooms, schools, families, and
communities. Teacher expectation effects are not universal; they vary by
teacher practices, student vulnerability, and contextual factors like
classroom composition. They also emphasize the importance of integrating
teacher expectation findings into teacher education. Teaching future
educators to avoid the negative effects of low expectations and to
provide appropriately challenging learning opportunities could foster
greater equity in student outcomes.

\begin{quote}
{\#yourturn}

Can the Pygmalion effect apply to self-expectations? How might your own
beliefs about your abilities influence your performance in a given task
or goal?
\end{quote}

\section{Conclusion}\label{conclusion}

The Pygmalion Effect is a social psychological phenomenon in which
higher expectations from others lead to improved performance. This
effect was first demonstrated by Robert Rosenthal and Lenore Jacobson in
their seminal 1968 study, \emph{Pygmalion in the Classroom}
(\citeproc{ref-Rosenthal_Jacobson_1968}{Rosenthal and Jacobson 1968}),
which showed that teachers' beliefs about students' potential could
significantly influence academic outcomes. While the original study laid
the groundwork for understanding this phenomenon, the decades of
subsequent research have added nuance to our understanding. Teacher
expectations can indeed enhance or hinder students' academic
achievements, but these effects are not uniform; they depend on various
factors, including teacher practices, student background, and the
context within which they operate.

\bookmarksetup{startatroot}

\chapter{\texorpdfstring{{Ego
Depletion}}{Ego Depletion}}\label{ego-depletion}

{written by Hannah Baumgart (original draft) and Rima-Maria Rahal
(revision)}

\section{The Classic}\label{the-classic-1}

Ego depletion is a social psychological concept that describes the
depletion of individuals' self-regulatory resources. Baumeister et al.
(\citeproc{ref-baumeister1998}{1998}) were the first to demonstrate ego
depletion effects in four different experimental settings: After having
to engage in an act of self-control (compared to a control task that
does not require self-control), willpower is used up and could not be
deployed as effectively in a subsequent task.

\begin{quote}
\phantomsection\label{def-egodepletion}{\#definition} Ego Depletion

A concept that describes willpower as a limited resource that can be
used up (depleted).
\end{quote}

In Experiment 1, the focus was on the act of resisting a temptation,
which requires self-control. Participants were randomly assigned to
different food conditions, by which the independent variables were
manipulated: Chocolate chip cookies and chocolate, radishes or no food
at all (control group). Participants in the radish control condition
were instructed to resist the tempting chocolates and instead eat
several the radishes that were laid out next to the chocolate. In the
chocolate condition, participants were asked to eat several cookies or
chocolates, which were laid out next to the radishes -- a task that was
not supposed to require much self-control. The actual intention behind
the experiment, to demonstrate ego depletion, was disguised with a cover
story to make sure participants would not get suspicious. They were told
the experiment was about taste perception.

\begin{quote}
{\#yourturn}

Which other tasks in your daily life require more or less willpower?
\end{quote}

In the no-food control condition, participants were not asked to taste
any food, but worked on the rest of the experiment.

After the participants had completed the willpower task resisting the
temptation of the foods presented to them, they had to complete
questionnaires on mood and restraint. Then they had to work on
``solving'' a problem-solving task, which was actually unsolvable. Here,
the time spent on trying to solve the problem before giving up was the
dependent variable.

The results showed significant differences between the three conditions,
with participants in the radish condition stopping earlier than those in
the chocolate or no-food condition. In conclusion, it was suggested that
craving chocolate but choosing to eat radishes depleted an internal
resource, leaving individuals less able to persist while trying to solve
the puzzles afterwards.

\begin{quote}
{\#yourturn}

If willpower can be depleted, how can it be ``refilled'' or built up
again?
\end{quote}

\section{The Aftermath}\label{the-aftermath-1}

Since this study, several hundred follow-up studies, including several
multi-lab studies that aimed to replicate the overall finding
(\citeproc{ref-hagger2010}{Hagger et al. 2010};
\citeproc{ref-vohs2021}{Vohs et al. 2021}) and several meta-analyses
(\citeproc{ref-hagger2010}{Hagger et al. 2010};
\citeproc{ref-carter2014}{Carter and McCullough 2014};
\citeproc{ref-dang2017a}{Dang 2017};
\citeproc{ref-bluxe1zquez2017}{Blázquez, Botella, and Suero 2017}) have
been carried out.

\begin{quote}
\phantomsection\label{def-multilabstudy}{\#definition} Multi-Lab Study

A research project in which researchers working at several different
locations (laboratories) implement the same experimental design and then
analyse the data together.
\end{quote}

These studies yielded mixed results, with some concluding that it was
highly unlikely that the ego depletion phenomenon does not exist (e.g.,
\citeproc{ref-hagger2010}{Hagger et al. 2010}), while others failed to
establish the effect despite relying on data from more than 2000
participants (e.g., \citeproc{ref-hagger2010}{Hagger et al. 2010}).
Publication bias has been argued to be high in the literature on ego
depletion (\citeproc{ref-inzlicht2015}{Inzlicht, Gervais, and Berkman
2015}), casting doubt on the effect.

\begin{quote}
{\#definition} Publication Bias

A tendency for research in line with established theories or showing
significant results to be more easily publishable than deviating
research.
\end{quote}

Continued research interest on ego depletion has brought forward varying
hypotheses regarding circumstances under which the effect might be
demonstrable and robust. The meta-analysis on ego depletion conducted by
Dang (\citeproc{ref-dang2017a}{2017}) investigated only studies with
sufficient initial effort exerted in the depleting, which was
hypothesized to lead to the ego depletion effect. The study ensured that
the depleting task required the use of self-control and excluded
manipulations that were less clearly related to self-control, such as
those based on social exclusion. Eight commonly used depletion tasks
were assessed in the meta-analysis: attention essay, attention video,
crossing out letters, emotion video, food trial, Stroop, thought
suppression, and working memory.

\begin{quote}
{\#yourturn}

Can you imagine what participants had to do in these tasks? Think about
a version of each task that would drain self-control and one that would
be less exhausting.
\end{quote}

The results showed that two of these exhausting tasks, attention video
and working memory, were not associated with significant changes in
subsequent self-control. Emotion videos, on the other hand, appeared to
be the most effective task and reduced subsequent self-control.

The overall analysis revealed a small to medium effect size for the ego
depletion effect. Correcting for publication bias, this effect was not
statistically significant when using the full sample of studies
identified. However, a separate analysis for reliable depletion tasks,
such as attention essay, emotion video and Stroop, showed the
significant effect remained when attempting to correct for publication
bias. This meta-analysis suggests that in special tasks, ego depletion
might occur, but that it is difficult to generalize to other
circumstances.

However, even in these special tasks, there is often no direct measure
of the initial depletion of willpower involved: manipulation checks on
whether willpower has been used up offer only an indirect measurement
(\citeproc{ref-friese2018}{Friese et al. 2018}).

\begin{quote}
{\#yourturn}

How could you objectiveley measure the amount of willpower available or
drained?
\end{quote}

\section{Conclusion}\label{conclusion-1}

The literature suggests a differentiated view on the potentially finite
nature of willpower is necessary (for a detailed overview, read more in
\citeproc{ref-friese2018}{Friese et al. 2018}). In the context of social
psychological theories, the ego depletion effect can be seen as an
important example of contradictory findings in research, where
publication bias may play a role. Although several hundred studies on
ego depletion have been published, we cannot be sure whether ego
depletion exists or not.

\begin{quote}
{\#yourturn}

Do you think ego depletion exists?
\end{quote}

The debate about ego depletion shows that individual findings should be
reassessed in several empirical demonstrations, including replication
attempts that can provide a more realistic picture of the effect or
construct. In this case, the original ego depletion effect may have been
initially inflated due to publication bias. Following closer
examination, it is less certain whether this effect indeed exists. The
example of the ego depletion literature also shows the importance of
examining the evidence closely, under the microscope, in order to ensure
that it meets the quality criteria that are essential for assessing
cumulative evidence of the overall effect.

\bookmarksetup{startatroot}

\chapter{\texorpdfstring{{Social
Facilitation}}{Social Facilitation}}\label{social-facilitation}

{written by Dearbhaile Vaughan (original draft), Kate Grady (original
draft), Cillian McHugh (revision), and Siobhán M. Griffin (revision)}

\section{The Classic}\label{the-classic-2}

Social Facilitation is a theory that posits that one will perform better
on a task when it is completed in the presence of others. In 1898,
Norman Triplett demonstrated that when people complete a task in
competition with another person, they perform better on the task
compared to completing the task alone
(\citeproc{ref-Triplett_1898}{Triplett 1898}). This seminal experiment
sparked a rich literature on the concept of ``social facilitation,'' a
term which was coined some 20 years later by Allport
(\citeproc{ref-Allport_1954}{1954}) -- the idea that the mere presence
of others can lead to improvements in performance
(\citeproc{ref-Aronson-et-al_2005}{Aronson, Wilson, and Akert 2005};
\citeproc{ref-Bond-Titus_1983}{Bond and Titus 1983}).

\begin{quote}
\phantomsection\label{def-socialfacilitation}{\#definition} Definition
of Social Facilitation

This theory proposes that the mere presence of others will positively
affect performance on a task.
\end{quote}

Triplett's seminal study (\citeproc{ref-Triplett_1898}{1898}) was the
result of perceived trends he observed in cyclists. Triplett noticed
that in both paced and competitive settings, cyclists tended to cycle
faster when accompanied by others. He had a multitude of theories as to
why this was, including both physical and psychological hypotheses. One
theory was that of the `Encouragement Theory', where the presence of a
friend would cheer on and ``keep the {[}participant's{]} spirits up.''
Other theories included: `Shelter Theory' -- the lead cyclist creates
shelter from the wind making it easier for those behind to cycle;
`Suction Theory' -- a vacuum is created by ``suction exertion'' from the
cyclist in front; and `Theory of Hypnotic Suggestions' -- that a
hypnosis effect is created by the wheels of the bicycle in front, and
this leads to better performance.

To test his theory, Triplett designed a lab-based study to examine if
the presence of a competitor stimulates competition arousal, which he
called ``dynamogenic factors.''

\begin{quote}
\phantomsection\label{def-dynamogenesis}{\#definition} Definition of
Dynamogenesis

An increase in the mental or motor activity of an already functioning
bodily system that accompanies any added sensory stimulation
(Merriam-Webster).
\end{quote}

For the experiment, two fishing reels were attached to a table to create
a type of pulley system that moved a flag around a four-metre course.
Children were invited to participate in this study. After a practice
period to allow children to become accustomed to the machine, they
completed six trials alternating between performing alone and performing
in competition with another child. There were rest periods in between
each trial to avoid the effects of fatigue. Performance was defined as
the time taken to complete one trial (four laps of the course) as
measured by a stopwatch. The results showed that children performed
better (i.e., completed the laps faster) during the competition/together
trials compared to the alone trials. However, some variation was noted
where some children, described as ``overstimulated,'' performed slower
on the together trials.

\begin{quote}
{\#yourturn}

Can you think of how the factors such as (i) age variability, (ii)
potential differences in practice times, (iii) lack of clarity around
the rest periods, and (iv) reporting on data from a subsample of 40
participants (out of 225) may have potentially affected the observed
findings?
\end{quote}

Triplett's findings and theory posited that competition stimulates
performance (competitive coaction), but subsequent researchers focused
on a broader application of this idea - that the mere presence of
another person would improve performance, competitive or not.

\section{The Aftermath}\label{the-aftermath-2}

Subsequent research focused on social facilitation across a number of
different social pressure contexts, including having an observer or
audience present, having an evaluative observer or audience, a
non-competing co-actor, and similar to Triplett's study - in the
presence of a competing co-actor (\citeproc{ref-Dashiell_1930}{Dashiell
1930}). Some research has highlighted the importance of task complexity.
For instance, Zajonc (\citeproc{ref-Zajonc_1965}{1965}) examined social
facilitation in a sample of cockroaches, showing that social presence
enhances performance on simple tasks but hinders performance on more
complex tasks (completing a runway vs completing a maze). Based on
behaviour theory (\citeproc{ref-Hull_1943}{Hull 1943};
\citeproc{ref-Spence_1956}{Spence 1956}), Zajonc postulated that
``generalized drive'' is what motivates habits. According to Zajonc's
theory, having other people around increases generalized drive, which
makes it easier for habitual dominant responses to occur. While for more
complex tasks the dominant response may not be the correct response
(\citeproc{ref-Bond-Titus_1983}{Bond and Titus 1983};
\citeproc{ref-Zajonc_1965}{Zajonc 1965})). However, a replication of
Zajonc's study did not fully replicate this effect; in simple and
complex tasks, the cockroaches performed more slowly when other
cockroaches were present (\citeproc{ref-Halfmann-et-al_2020}{Halfmann,
Bredehöft, and Häusser 2020}).

\begin{quote}
\phantomsection\label{def-generalizeddrive}{\#definition} Definition of
Generalized Drive

The presence of others leads to an increase in generalized drive, thus
facilitating habitualised dominant responses.
\end{quote}

Although Zajonc (\citeproc{ref-Zajonc_1965}{1965}) believed that the
mere presence of others is the necessary ingredient in producing social
facilitation effects, other researchers disagreed. Cottrell et al.
(\citeproc{ref-Cottrell_1972}{1972};
\citeproc{ref-Cottrell-et-al_1968}{1968}) argued that social
facilitation occurs when a third party is perceived to be observing the
performance, but that mere presence (without observation) was not
sufficient to produce social facilitation effects. It is the expectation
of evaluation that increases drive, and thus influences performance
(\citeproc{ref-Bond-Titus_1983}{Bond and Titus 1983};
\citeproc{ref-Cottrell_1972}{Cottrell 1972};
\citeproc{ref-Weiss_Miller_1971}{Weiss and Miller 1971}).

There are a number of theoretical explanations to explain how, why, and
when social facilitation effects occur (for reviews see
\citeproc{ref-Bond-Titus_1983}{Bond and Titus 1983};
\citeproc{ref-Seitchik-et-al_2017}{Seitchik, Brown, and Harkins 2017}).
Some key theories include:

\begin{itemize}
\tightlist
\item
  \emph{Distraction-conflict theory:} the idea that the presence of
  others is distracting and takes up attention resources which may lead
  to cognitive overload, reducing attention on the task
  (\citeproc{ref-Baron_1986}{Baron 1986};
  \citeproc{ref-Sanders-et-al_1978}{Sanders, Baron, and Moore 1978}).
  This may result in dominant responses facilitating performance when
  the task is simple and requires attention to a small number of cues,
  but when the task is more complex or demands attention to a larger
  number of cues performance may be hindered.
\item
  Muller and Butera's (\citeproc{ref-Muller_Butera_2007}{2007})
  \emph{Integrated distraction-conflict theory and Social comparison
  theory} (\citeproc{ref-Festinger_1954}{Festinger 1954}); e.g., that
  people compare their own skills to other people's skills), and
  proposed that when in a co-action setting people can experience
  self-evaluation threat which may increase their attentional focus, in
  particular when a co-actor is seen to be superior, increasing drive
  and thus performance.
\end{itemize}

\begin{quote}
\phantomsection\label{def-distractionconflicttheory}{\#definition}
Definition of Distraction-Conflict Theory

This theory states that attentional conflict, a type of response
conflict regarding what attentional response one should make, can arise
when the social presence of others (co-actors or an audience) is
distracting, at least when the task is attention demanding. The actor
may then be at risk of cognitive overload as a result of this conflict,
which would ultimately lead to a limitation in their ability to focus on
the task.
\end{quote}

\begin{quote}
\phantomsection\label{def-socialcomparisontheory}{\#definition}
Definition of Social Comparison Theory

According to the social comparison theory, people are motivated to
assess their own beliefs and skills by comparing them to external
images. These images can be comparisons to other people or a reference
to physical reality. Individuals have a tendency to view images
portrayed by others as accessible and realistic and subsequently make
comparisons between themselves, other people, and these idealized
images.
\end{quote}

\subsection{Practical Implications Arising from Triplett's Original
Study}\label{practical-implications-arising-from-tripletts-original-study}

Research on social facilitation effects has highlighted its practical
implications in real-world settings. For example, Anderson-Hanley,
Arciero, and Snyder (\citeproc{ref-Anderson-Hanley_et_al_2011}{2011})
demonstrated that adults riding on stationary bikes with virtual reality
equipment exercised more when a competitive fictional character was
introduced compared to cycling alone (but only if they scored highly on
self-reported competitiveness). Furthermore, people have been found to
consume more food (\citeproc{ref-deCastro_1994}{Castro 1994}), donate
more money (\citeproc{ref-Izuma-et-al_2010}{Izuma, Saito, and Sadato
2010}), and spend more money (\citeproc{ref-Sommer-et-al_1992}{Sommer,
Wynes, and Brinkley 1992}), when with other people compared to being
alone. However, sometimes the presence of another is seen to have
detrimental effects. For instance, an analysis of archival data
demonstrated that learner drivers who took their driving test with
another individual awaiting their test present were more likely to fail
than those who took the test without an observer
(\citeproc{ref-Rosenbloom-et-al_2007}{Rosenbloom et al. 2007}).

\begin{quote}
{\#yourturn}

Can you think of a time when you performed worse on a task because there
was another individual present? And can you think of a time when you
performed better on a task when there was another individual present?
\end{quote}

\subsection{A Reanalysis of Triplett's
Data}\label{a-reanalysis-of-tripletts-data}

Since 1898, more advanced statistical methods are now at researchers'
disposal. Strube (\citeproc{ref-Strube_2005}{2005}) reanalysed
Triplett's (\citeproc{ref-Triplett_1898}{1898}) data, exploring both
within-person differences in alone vs.~together conditions
(within-subjects tests) and differences between people across the alone
vs.~together trials (between-subjects tests). This re-analysis
demonstrated that in general performance in the competition trials was
better than the alone trials (between-subjects test); however, this was
not a statistically significant difference. Likewise, looking at
within-participant variation, there was only a marginally significant
effect for performing better on the competition trials compared to a
person's alone trials.

\subsection{Replication of the Original
Study}\label{replication-of-the-original-study}

A recent pre-registered study directly replicated Triplett's
(\citeproc{ref-Triplett_1898}{1898}) original experiment, addressing
some of the limitations mentioned earlier -- namely the small (and
underpowered) sample size, standardization of experimental trials and
rest periods, as well as examining if gender moderated the effects
(\citeproc{ref-McHugh_et_al_2025}{McHugh et al. 2025}). This analysis of
\textgreater400 children aged 7-13 years, who were age- and
gender-matched, demonstrated that participants completed the task
quicker during the together trials compared to the alone trials. Gender
moderated this effect, with females completing the task faster on
average, and the social facilitation/competitive co-action effect was
stronger for females. This replication provides support for Triplett's
original findings.

\section{Conclusion}\label{conclusion-2}

Overall, it appears that in some settings the presence of another
(whether evaluative or non-evaluative, or co-actor, competitor or
observer) affects performance. Often the presence of another appears to
facilitate performance or dominant response tendencies, but the
conditions under which this occurs need further examination as sometimes
the presence of another hinders performance. While Triplett focused on
competitive coaction effects, which was later termed social facilitation
(and gave rise to this literature), it is important to note that the
theory of social facilitation relates to mere presence of another
individual affecting performance. Triplett's
(\citeproc{ref-Triplett_1898}{1898}) experiment and the more recent
replication (\citeproc{ref-McHugh_et_al_2025}{McHugh et al. 2025}) are
not able to disentangle if the effects on performance are truly due to
mere presence of another person (i.e., social facilitation) or due to
competition.

\begin{quote}
{\#yourturn}

Why do you think it matters whether performance depends on mere presence
of others or if others need to be co-actors and/or competitors?
\end{quote}

More research is needed to fully understand what is driving the observed
effects. Think back to the practical implications section of this
chapter. If we know under what conditions mere presence affects
performance (positively and negatively), or under what conditions having
someone engaged in the same task as us (co-actor) or even competing
against us, then this can help us design optimal environments for a
range of performance-based activities, such as learning and
exercise/sport.

\bookmarksetup{startatroot}

\chapter{\texorpdfstring{{Intergroup Contact
Theory}}{Intergroup Contact Theory}}\label{intergroup-contact-theory}

{written by Vanessa Müller (original draft), Milica Ninković (revision),
Raul Szekely (revision), and Lukas Wallrich (revision)}

\section{The Classic}\label{the-classic-3}

In the mid-20th century, after the horrors of World War II and during
fights against official racial segregation, social scientists began
asking a deceptively simple question: If you bring members of
conflicting groups into contact, will they start to get along? Opinions
were divided. Some warned that interracial contact would only breed
``suspicion, fear, resentment, disturbance, and at times open conflict''
(\citeproc{ref-baker_negro-white_1934}{Baker 1934}, pg. 120; cited in
\citeproc{ref-Pettigrew-Tropp_2006}{Pettigrew and Tropp 2006};
\citeproc{ref-brophy_luxury_1945}{I. N. Brophy 1945}). Others were more
optimistic, suggesting that isolation allowed prejudice to ``grow like a
disease'' (\citeproc{ref-brameld_minority_1946}{Brameld 1946}, pg. 245;
cited in \citeproc{ref-Pettigrew-Tropp_2006}{Pettigrew and Tropp 2006})
and that, under the right circumstances, interaction could lead to
``mutual understanding and regard''
(\citeproc{ref-lett_techniques_1945}{Lett 1945}, pg. 35; cited in
\citeproc{ref-Pettigrew-Tropp_2006}{Pettigrew and Tropp 2006}). This
debate set the stage for one of social psychology's most influential
ideas: intergroup contact theory.

One of the first real-world tests of these ideas came in the United
States Merchant Marine shortly after World War II. In 1946, sociologist
Norman Brophy surveyed white sailors now serving in newly desegregated
ship crews (\citeproc{ref-brophy_luxury_1945}{I. N. Brophy 1945}). He
created a ``prejudice index'' from interview questions and looked for
patterns. Expected predictors of racial attitudes -- such as where a
sailor was born or how much education he had -- turned out not to matter
much. Instead, direct personal contact was the standout factor. Brophy
found that white seamen who had never shipped with a Black crewmate
scored highest in prejudice, whereas those who had taken four or more
voyages with Black crewmates scored the lowest. In the cramped,
cooperative environment of a ship -- an ``artificial society'' where
survival depended on teamwork -- many sailors discovered they could no
longer ``afford the luxury'' of prejudice. And Brophy wasn't alone in
this observation. Similar studies, mostly in the United States, showed
more positive attitudes among White police officers who worked with
Black colleagues (\citeproc{ref-kephart_racial_1957}{Kephart 1957}), and
White residents who lived in mixed buildings where they had the
opportunity to interact with Black neighbours
(\citeproc{ref-deutsch_interracial_1951}{Deutsch and Collins 1951}).
These early findings suggested that prejudice was not immutable, but
could change with contact.

\begin{quote}
\phantomsection\label{def-prejudice}{\#definition} Prejudice

A negative attitude toward a group and its members, often based on
stereotypes rather than direct experience.
\end{quote}

These patterns spurred social scientists to theorise why and when
contact might reduce prejudice. In his landmark book \emph{The Nature of
Prejudice} (\citeproc{ref-Allport_1954}{1954}), the psychologist Gordon
Allport proposed what has become known as the contact hypothesis: the
idea that under appropriate conditions, interpersonal contact between
members of different groups can be one of the most effective ways to
reduce intergroup prejudice. Crucially, Allport
(\citeproc{ref-Allport_1954}{1954}) did not claim that contact always
works. Instead, he specified four optimal conditions that, in his view,
were needed for contact to reduce prejudice:

\begin{enumerate}
\def\labelenumi{\arabic{enumi}.}
\tightlist
\item
  \textbf{Equal Status:} The groups should have equal status within the
  contact situation.
\item
  \textbf{Common Goals:} The groups should strive towards a mutually
  beneficial outcome.
\item
  \textbf{Cooperation (Not Competition):} The interaction should require
  cooperative effort from members of different groups.
\item
  \textbf{Support of Authorities or Norms:} The contact experience
  should have the explicit or implicit support of authorities, law, or
  social norms (e.g., teachers who encourage intergroup exchange
  explicitly).
\end{enumerate}

Allport (\citeproc{ref-Allport_1954}{1954}) hypothesised that when these
conditions are met, contact encourages people to view one another as
individuals and teammates, and thus perceive members of the ``other''
group as part of a shared ``us'' rather than a separate ``them.''
Interpersonal contact could then reduce ignorance and anxiety, increase
empathy and understanding, and ultimately chip away at prejudice
(\citeproc{ref-Allport_1954}{Allport 1954}). On the other hand, Allport
(\citeproc{ref-Allport_1954}{1954}) warned that contact in unfavourable
circumstances could backfire.

\begin{quote}
{\#yourturn}

Think about a common intergroup contact situation in your community (for
example, students from different backgrounds meeting at university, or
neighbours from different ethnic groups interacting). Does that
situation meet Allport's four optimal conditions (equal status, common
goals, cooperation, and supportive norms)? How might the presence or
absence of these conditions be influencing how well the groups get
along?
\end{quote}

\section{The Aftermath}\label{the-aftermath-3}

Allport's (\citeproc{ref-Allport_1954}{1954}) formulation of the contact
hypothesis was hugely influential. It inspired a wave of research from
the 1950s onward as psychologists, sociologists, and others researched
the power of contact in a variety of groups and settings, mostly in
observational research. By the turn of the 21st century, the evidence
base had become enormous -- though somewhat scattered. Hundreds of
studies across dozens of countries and intergroup contexts had examined
intergroup contact in one form or another, and the contact hypothesis
had become a cornerstone of social psychology. The overarching question
remained: Does contact typically work to reduce prejudice, and under
what conditions?

\begin{quote}
\phantomsection\label{def-observationalresearch}{\#definition}
Observational Research

A study design where researchers measure variables as they naturally
occur, without manipulating them. Observational studies can reveal
associations between variables but cannot, on their own, establish that
one causes the other.
\end{quote}

\subsection{The Classic Meta-Analysis}\label{the-classic-meta-analysis}

By the early 2000s, it was challenging to see the big picture in contact
research. To address this, psychologists Thomas Pettigrew and Linda
Tropp (\citeproc{ref-Pettigrew-Tropp_2006}{2006}) conducted a landmark
quantitative review. In 2006, they published a meta-analysis
synthesising findings from 515 studies (covering 713 independent samples
and over 250,000 participants) that had studied intergroup contact.
Across this vast body of work, they found a consistent pattern: people
who reported more positive contact with members of an outgroup also
tended to report lower levels of prejudice toward that group.

\begin{quote}
\phantomsection\label{def-meta-analysis}{\#definition} Meta-analysis

A statistical technique that combines the results of multiple
independent studies to estimate an overall effect. Meta-analyses can
reveal patterns across a large body of research, but the quality of
their conclusions depends on the quality and comparability of the
included studies.
\end{quote}

Pettigrew and Tropp (\citeproc{ref-Pettigrew-Tropp_2006}{2006})
concluded that ``intergroup contact can promote reductions in
prejudice'' (p.~751) and that ``there is little need to demonstrate
further contact's general ability to lessen prejudice'' (p.~766), even
in situations when not all optimal conditions are met. The average
effect size was substantial by social science standards (Cohen's d ≈
0.43). With this uplifting message, their meta-analysis has become one
of the most-cited papers in social psychology, with over 13,000
citations to date.

However, most of the studies they synthesised were observational rather
than experimental, meaning they measured naturally occurring contact
rather than manipulating it. While observational studies are valuable
for spotting consistent relationships, they cannot, on their own,
establish that contact caused the reduction in prejudice. For that,
experiments are usually needed, and only 5\% of the studies in the
meta-analysis are true experiments. Pettigrew and Tropp
(\citeproc{ref-Pettigrew-Tropp_2006}{2006}) acknowledged this limitation
but advanced various arguments why their results still indicate causal
effects. Most importantly, studies that used more rigorous methods (for
example, longitudinal designs or experiments) tended to find larger
effects of contact than weaker, correlational studies did.

Pettigrew and Tropp (\citeproc{ref-Pettigrew-Tropp_2006}{2006}) also
aimed to assess whether the benefits of contact generalise -- that is,
does having a positive experience with, say, one Black teammate make a
white person feel more positively toward Black people in general?
Encouragingly, many studies did find evidence of generalisation:
improved attitudes often extended beyond the specific individuals
involved to the outgroup as a whole. For example, if a white student
befriended a Latino roommate, not only might their attitude toward that
roommate improve, but their overall attitude toward Latinos could become
more favourable as well. This kind of generalisation is crucial if
contact is to have a broad social impact, and the meta-analysis
indicated that it often occurs.

For a time, Pettigrew and Tropp's
(\citeproc{ref-Pettigrew-Tropp_2006}{2006}) comprehensive review seemed
to settle the debate: Intergroup contact works. With so many studies and
an authoritative meta-analysis affirming that contact typically reduces
prejudice (even outside of perfect conditions), the contact hypothesis
gained even more prominence. Textbooks began to state confidently that
positive contact is a proven method to improve intergroup relations.
However, the story didn't end there. Sceptics and careful scientists
raised important questions and cautions that would spark the next wave
of investigations. Most importantly: is the evidence causal? If we
observe that people who have more friends from other groups also show
lower prejudice, it's not always clear which way the arrow of causality
points -- does contact reduce prejudice, or do less-prejudiced people
simply seek out more contact? Pettigrew and Tropp's
(\citeproc{ref-Pettigrew-Tropp_2006}{2006}) analysis went a long way
toward addressing this by showing that the best studies (including
experiments) found stronger effects, but still, the bulk of studies in
their database were not true experiments. Additionally, critics wondered
about unpublished null findings: were there ``file drawer'' studies
where contact had no effect that were never known, potentially making
the published literature look overly rosy? These cautions set the stage
for a new generation of research that aimed to more robustly test when
and how contact works -- and to probe its limits.

\begin{quote}
{\#yourturn}

Why is it important to go beyond correlational evidence (where we simply
observe relationships) when evaluating whether intergroup contact truly
reduces prejudice? What kinds of studies or methods would give more
convincing evidence of causation?
\end{quote}

\subsection{New Insights and Challenges: Refining the
Theory}\label{new-insights-and-challenges-refining-the-theory}

By the 2010s, researchers began responding to these methodological
concerns, bringing fresh scrutiny to the study of intergroup contact.
For instance, a review by Elizabeth Paluck, Seth Green, and Donald Green
(\citeproc{ref-Paluck-et-al_2019}{2019}) specifically re-evaluated the
contact hypothesis from a rigorous causal perspective. They exclusively
focused on studies that met a high bar for evidence: field experiments
with random assignment to a contact condition versus a control
condition, and outcome measures assessed after the contact experience
was concluded. Out of the thousands of contact studies conducted over
the decades, Paluck, Green, and Green
(\citeproc{ref-Paluck-et-al_2019}{2019}) found only 27 experiments that
fit these strict criteria up to that point. (Notably, almost two-thirds
of those 27 had been published after Pettigrew and Tropp's
(\citeproc{ref-Pettigrew-Tropp_2006}{2006}) meta-analysis, reflecting
the field's recent push for experimental work.)

\begin{quote}
\phantomsection\label{def-experiment}{\#definition} Experiment

A study where researchers deliberately manipulate one or more variables
and randomly assign participants to different conditions. Random
assignment helps ensure the groups are similar before the intervention,
so differences in outcomes are more likely to be caused by the
manipulation rather than by pre-existing differences.
\end{quote}

The good news was that, overall, the evidence from these rigorously
controlled studies still supported Pettigrew and Tropp's
(\citeproc{ref-Pettigrew-Tropp_2006}{2006}) basic conclusion: intergroup
contact ``typically reduces prejudice.'' In their meta-analysis of the
27 experiments, Paluck, Green, and Green
(\citeproc{ref-Paluck-et-al_2019}{2019}) found that the average effect
of being randomly assigned to a positive contact experience was a
reduction in prejudice levels compared to the control groups, with
Cohen's d ≈ 0.39, very similar to Pettigrew and Tropp's
(\citeproc{ref-Pettigrew-Tropp_2006}{2006}) result. This helps rebut the
idea that the contact-prejudice link was merely a selection effect; even
when people were assigned to have contact, prejudice tended to go down,
on average. However, the experimental evidence also revealed some
important caveats. One striking finding was that contact's effectiveness
varied considerably by context and target group. In particular,
interventions aimed at reducing ethnic or racial prejudices (for
example, between Israelis and Palestinians, or between white and Black
Americans) tended to show weaker effects than interventions aimed at
reducing prejudice toward other stigmatised groups (such as people with
disabilities or members of an opposing political party). In other words,
contact worked least well for some of the most historically entrenched
divides like race and ethnicity. On the flip side, contact interventions
addressing prejudices that might be less emotionally charged or less
tied to deep-rooted group identities (for example, toward the disabled,
or between fans of rival sports teams) produced relatively larger
improvements on average.

Paluck and colleagues (\citeproc{ref-Paluck-et-al_2019}{2019} ) also
highlighted critical gaps in the evidence. For example, they found an
almost complete lack of field experiments focused on adult populations
dealing with racial or ethnic prejudice -- the context the contact
hypothesis had originally been about and arguably still one of the most
important areas for policy. Additionally, very few studies had
systematically tested Allport's (\citeproc{ref-Allport_1954}{1954})
optimal conditions by manipulating those factors to see which mattered
most. The authors concluded that these gaps need to be filled before we
can confidently advise policymakers to rely on contact to remedy
societal prejudice. In short, their message was not ``contact doesn't
work'' but rather ``contact can work, but we need better evidence,
especially on the toughest cases and the crucial conditions, to
understand how to use it most effectively.''

\begin{quote}
{\#yourturn}

Intergroup contact seems to yield larger prejudice reductions for some
kinds of group differences (for instance, attitudes toward people with
disabilities) than for others (like attitudes between ethnic groups).
Why do you think this might be? Consider the nature of prejudice or
anxiety in each case. What factors could make prejudice based on
race/ethnicity harder to change through contact compared to prejudice
toward people with disabilities, and vice versa?
\end{quote}

\subsection{An Outstanding Modern Study: Contact on the Soccer Field in
Post-ISIS
Iraq}\label{an-outstanding-modern-study-contact-on-the-soccer-field-in-post-isis-iraq}

To illustrate both the strengths and limitations of intergroup contact
in action, consider a modern field experiment that put Allport's
(\citeproc{ref-Allport_1954}{1954}) hypothesis to a challenging test.
Political Scientist Salma Mousa conducted a remarkable study in
post-conflict Iraq, published in 2020, to see if positive contact could
help heal rifts between deeply divided religious communities
(\citeproc{ref-mousa_building_2020}{Mousa 2020}). The setting was
Northern Iraq in the aftermath of the ISIS terror reign. In 2014, ISIS
had overrun the region, committing atrocities including the displacement
of almost the entire Christian population from certain towns. By 2016,
after ISIS was defeated, many displaced Christian families began
returning to their hometown of Qaraqosh, a historically Christian town
that had been scarred by violence. These returning Christians carried
intense distrust and resentment toward the local Muslims. The Christians
feared that some Muslim neighbours had been complicit with ISIS, or at
least did not suffer as they had, and rumours and grievances ran
rampant. In turn, Muslim residents felt unwelcome and resented the
suspicions. In this tense post-ISIS context, the two groups lived
segregated lives, with social contact minimal and fraught. Prejudice and
fear were high on both sides.

Mousa (\citeproc{ref-mousa_building_2020}{2020}) wondered if a carefully
designed contact intervention could begin to rebuild trust and
coexistence in this environment. She chose a grassroots approach:
recreational soccer teams. Why soccer? Importantly, soccer in this
context naturally met many of Allport's
(\citeproc{ref-Allport_1954}{1954}) optimal conditions for positive
contact. For one, players on a team share a common goal -- to win
matches -- and must cooperate closely to do so (passing the ball,
strategising, etc.). Team sports also tend to equalise status; when
everyone puts on the same jersey, they have equal status as teammates on
the field. Additionally, Mousa
(\citeproc{ref-mousa_building_2020}{2020}) worked with local
organisations and community leaders (including church officials) to
support and endorse the league, lending authority approval to this
intergroup activity. In short, the intervention was deliberately
structured to tick all of Allport's (\citeproc{ref-Allport_1954}{1954})
boxes.

Here's how the experiment worked. Mousa
(\citeproc{ref-mousa_building_2020}{2020}) invited young Christian men
in Qaraqosh who were interested in playing soccer to form teams in a new
reconciliation soccer league. These men formed teams mostly with friends
or neighbours, so initially, all-Christian teams. The twist was that
Mousa (\citeproc{ref-mousa_building_2020}{2020}) then randomly assigned
half of the league's teams to receive several Muslim players as
additions to their roster (the other half of the teams remained
all-Christian and served as a control group). The Muslim players were
recruited from outside the town (from camps of displaced Muslims nearby)
and chosen to be of similar skill level to the Christian players, so
that they could genuinely contribute on the field without dominating or
being token outsiders. In total, each ``mixed'' team got three Muslim
teammates added. All teams -- mixed and all-Christian alike -- then
played in the same 8-week amateur league, facing each other in matches.
Importantly, every other aspect of the league was the same for everyone:
all teams had the same equipment, schedule, and participated under the
same community-endorsed conditions, with the only difference being
whether your teammates included Muslims or not. This experimental setup
meant that if differences emerged between players on mixed teams versus
all-Christian teams, the only systematic explanation would be the
experience of having (or not having) Muslim teammates.

At first, the intervention faced friction. Some Christian players were
unhappy about Muslims joining their teams. In the early weeks, there
were incidents of mistrust and even hostility -- for example, a few
Christian team members openly told the organisers ``We don't want
Muslims; they will ruin the league.'' Such remarks underscored just how
deep the suspicion ran in this community; it wasn't an easy start. But
as the season progressed and these young men practised and competed
side-by-side, the tone began to shift. By about the mid-point of the
season, signs of camaraderie had emerged. One small episode stood out:
when some Christian players learned that their new Muslim teammates were
struggling to afford taxi fare to the games (travelling from a distant
displacement camp), the Christian players pooled money to help cover the
cost so their teammates could make it to matches. On the field,
teammates started to celebrate goals together and encourage one another.
Over time, a shared team identity -- we are the Lions, we are teammates
-- began to form, overlaying the previous religious divide. A Christian
player, asked later about his experience, reflected that ``I learned
that Muslims could be friends of ours, even like brothers.'' The
transformation was not instant or universal, but by the end of the
league, many of the initial anxieties had given way to friendly
competition and mutual respect on these mixed teams.

So, did this Allportian (\citeproc{ref-Allport_1954}{Allport 1954})
contact experience actually change attitudes or behaviors? Mousa's
(\citeproc{ref-mousa_building_2020}{2020}) results were revealing. They
showed both encouraging positive outcomes and clear limits. First,
consider the effects within the context of the league itself -- that is,
how the Christian players felt and acted toward their Muslim teammates
(and other Muslims in the league): The Christian players who had Muslim
teammates ended up displaying significantly more positive behaviours
toward Muslim peers compared to players on all-Christian teams. For
example, at the end of the season, each team voted for a member of an
opposing team to receive a sportsmanship award. Christians on mixed
teams were more than 15 percentage points more likely to vote for a
Muslim player (from another team) for this award than were Christians on
all-Christian teams. This indicated greater esteem and fairness toward
Muslim peers. Moreover, when sign-ups opened for a new season, the
mixed-team Christians were much more willing to play on a mixed team
again -- they registered at higher rates for a subsequent mixed league
-- whereas many all-Christian team players declined to sign up once they
heard teams might be mixed. Perhaps most impressively, about six months
after the experiment, Mousa (\citeproc{ref-mousa_building_2020}{2020})
found that many of the mixed-team players were still regularly meeting
up with their former Muslim teammates to practice together and maintain
their friendship. In fact, roughly one-third of the mixed teams
continued to meet socially for pick-up soccer games long after the
official league ended, whereas almost none of the all-Christian teams
chose to continue gatherings that included outgroup members. These
findings show that meaningful friendships and trust did form through the
contact intervention. By all accounts, prejudice had decreased, at least
with respect to those specific Muslim teammates and other known Muslim
players.

However, now consider what happened outside the context of the league --
in the broader community and in attitudes toward Muslims in general.
Here, the findings were more sobering: The positive effects of contact
did not substantially generalise to Muslims beyond those directly
encountered. In surveys and behavioural measures after the season,
Christian participants who had played with Muslim teammates showed no
significant change in their willingness to interact with unknown Muslims
or visit Muslim communities compared to the control group. For instance,
having had Muslim teammates generally did not make Christian players
more likely to say they would patronise a restaurant in a nearby
majority-Muslim city, nor did it increase their attendance at a mixed
social event in town. When asked about broader attitudes, those who
experienced contact did express somewhat stronger abstract support for
coexistence or the idea that Christians and Muslims could be friends,
but their core beliefs about Muslims as a group (for example, levels of
trust toward Muslim strangers or stereotypes about Muslims) remained
essentially as negative as before. In Mousa's
(\citeproc{ref-mousa_building_2020}{2020}) own words, while the
Christian players found it possible to trust and befriend specific
Muslim individuals they got to know, extending trust to Muslim strangers
outside that circle was ``too much of an ask'' in the aftermath of war.
In short, the contact intervention succeeded in forging new cross-group
friendships and improving attitudes toward those individuals, but it
largely failed to shift the participants' generalised feelings about the
outgroup as a whole or their behaviour in other contexts.

This pattern -- friendships without broad reconciliation -- highlights a
crucial challenge for intergroup contact theory. Mousa's
(\citeproc{ref-mousa_building_2020}{2020}) study offers an inspiring
proof-of-concept that even in a highly fraught, post-conflict setting, a
well-designed contact program, featuring Allport's
(\citeproc{ref-Allport_1954}{1954}) optimal conditions, can produce
genuine goodwill and cooperation between former adversaries. The fact
that young men who initially hated the idea of playing with ``the
other'' ended up forming lasting bonds is powerful. It shows that under
the right conditions, enemies can indeed become teammates, even friends.
On the other hand, the limited scope of these changes tempers the
optimism. The contact in this study changed how people felt about
particular outgroup members, but not necessarily about the outgroup at
large. From a policy or peacebuilding standpoint, that is a big
limitation: improving one-to-one relationships is wonderful for those
individuals, but it may not significantly mend the overall social fabric
or reduce the kind of generalised fear that fuels wider conflict.
Mousa's (\citeproc{ref-mousa_building_2020}{2020}) findings align with
what many other studies have found and what is now a central puzzle in
contact research -- the generalization problem. How can we ensure that
the effects of contact spread beyond the immediate participants and
influence attitudes more broadly? If positive contact only affects the
small circle of people directly involved, its ability to reduce
community-wide prejudice or conflict is limited.

\begin{quote}
{\#yourturn}

In the soccer study, Christian players clearly grew more accepting of
the Muslim teammates they got to know personally, yet their attitudes
toward Muslim strangers remained unchanged. Why do you think a positive
experience with a few individuals might fail to generalise to the entire
outgroup? What psychological factors might be at play? Can you think of
any additional measures or tweaks to the intervention that might help
encourage broader changes in attitudes or trust (for example, activities
that mix the groups in other settings, discussions that address group
stereotypes, etc.) to help bridge that gap?
\end{quote}

Mousa's (\citeproc{ref-mousa_building_2020}{2020}) soccer experiment
encapsulates both the promise and the limitations of intergroup contact.
It provides a vivid example that contact can work -- even under pretty
challenging conditions, it built trust and friendship where there was
initially fear and hostility. At the same time, it underscores that a
single intervention, even a well-crafted one, is no panacea for deeply
rooted prejudices. Especially in contexts of recent violence and trauma,
biases may run so deep that it takes much more than a brief intervention
to budge generalised attitudes. These nuanced outcomes have prompted
researchers to investigate strategies to amplify and extend contact
effects. How might we design contact interventions that not only improve
attitudes toward the people directly involved, but also shift
perceptions of the broader group? This remains an active area of
research.

In fact, as the field has progressed, experts have adopted a more
cautious tone about what contact can realistically achieve on its own.
In 2021, Paluck et al. (\citeproc{ref-paluck_prejudice_2021}{2021})
published an extensive review of 418 prejudice-reduction experiments
conducted between 2007 and 2019, a collection that included many
contact-based interventions alongside other approaches. The results of
this review were mixed and somewhat concerning. On one hand, many of the
experiments reported at least some positive effects on attitudes,
suggesting there are reasons for optimism. On the other hand, the
authors uncovered ``troubling indications of publication bias,'' meaning
that studies showing big success were likely overrepresented in the
literature, while those with null or tiny effects may not have been
published. When they statistically accounted for this bias, the overall
picture became less rosy. Furthermore, three-quarters of interventions
in that review were very ``light-touch'' or brief, such as a short
workshop, a single encounter, or a one-time media exposure. Not
surprisingly, any positive changes from such brief interventions often
faded over time or were quite limited in scope. In the relatively few
cases where more intensive, long-term interventions were implemented
(what the authors called ``landmark studies''), the effects on prejudice
tended to be modest at best. This included some multi-week educational
programs, extended intergroup dialogues, and other sustained efforts --
many showed only small improvements, highlighting how stubborn
prejudices can be. Paluck-et-al\_2021 concluded that new theoretical
innovation is needed to achieve larger and more lasting impacts. They
suggested that perhaps contact on its own is often too limited, and that
combining contact with other approaches (or addressing larger structural
issues in tandem) might be necessary to produce more substantial change.
In their view, simply throwing diverse people together for a short
period is rarely a magic fix; researchers need to think bigger about the
mechanisms of change and consider multi-pronged solutions.

Most recently, the strongest tests of the contact hypothesis have been
compiled in a 2025 meta-analysis by economist Matt Lowe
(\citeproc{ref-lowe_has_2025}{Lowe 2025}). Lowe
(\citeproc{ref-lowe_has_2025}{2025}) focused exclusively on the
highest-quality studies: those that were pre-registered, randomised
experiments on intergroup contact.

\begin{quote}
\phantomsection\label{def-pre_registered_study}{\#definition}
Pre-Registered Study

A study in which the researchers publicly register their hypotheses,
methods, and analysis plan before collecting data. Pre-registration
helps increase transparency and credibility -- it prevents researchers
from changing their analyses or selecting results after seeing the data,
which can lead to false-positive findings.
\end{quote}

By zeroing in on these rigorously planned studies, Lowe
(\citeproc{ref-lowe_has_2025}{2025}) aimed to eliminate biases
introduced by practices like p-hacking or cherry-picking of data --
practices that can inflate apparent effects.

\begin{quote}
\phantomsection\label{def-p_hacking}{\#definition} p-hacking

The practice of misusing data analysis to find patterns that can be
presented as statistically significant, often by trying many variable
combinations or statistical tests until something ``significant'' turns
up. This can lead to unreliable conclusions because it capitalises on
chance patterns in the data.
\end{quote}

\begin{quote}
\phantomsection\label{def-cherry_picking}{\#definition} Cherry-Picking

Reporting only the data, outcomes, or time frames that support one's
hypothesis while ignoring or dismissing those that do not. This makes
the story or articles simpler and might make them more publishable, but
provides a distorted view of the evidence.
\end{quote}

The findings are instructive. When considering only these
methodologically pristine studies, the average effect of intergroup
contact on prejudice outcomes was much smaller than earlier reviews had
suggested, with d ≈ 0.1, a quarter of the effect size suggested by
Pettigrew and Tropp (\citeproc{ref-Pettigrew-Tropp_2006}{2006}). In
plain language, this means that the effect is statistically significant,
but on average, quite modest. Lowe's
(\citeproc{ref-lowe_has_2025}{2025}) meta-analysis also reinforces the
now-familiar theme about specificity vs.~generality: contact's benefits
tend to be localised. People's attitudes and behaviours toward the
particular individuals they met often improved more strongly than their
attitudes toward the outgroup in general. Broad attitude change was much
less common, with many studies finding little to no shift in generalised
prejudice or policy views even when interpersonal warmth increased. That
contact's effects often fail to generalise widely is now recognised as
one of the central challenges in the field.

\section{Conclusion}\label{conclusion-3}

Seven decades after Allport (\citeproc{ref-Allport_1954}{1954}) first
set out the contact hypothesis, it remains a cornerstone of
prejudice‐reduction research. From post‐war merchant ships and
integrated housing projects to modern field experiments, the idea has
consistently shaped both science and policy. Its core message that
prejudice is not fixed and can change through structured, positive
interaction helped shift thinking away from segregation toward
integration as a deliberate tool for improving relations.

The evidence, however, shows that contact is no cure‐all. Gains are
often local, improving attitudes toward specific individuals but failing
to generalise to the wider group. Outcomes depend heavily on context,
structure, and the quality of interaction. For that, Allport's
(\citeproc{ref-Allport_1954}{1954}) optimal conditions (equal status,
common goals, cooperation, and authority support) remain a useful guide,
though further research is needed. Contact is also not always positive,
and researchers have started taking that more seriously, as negative
experiences can be as powerful, if not more so, than positive ones,
though they are fortunately rare
(\citeproc{ref-paolini_negativity_2024}{Paolini et al. 2024}). Contact
also needs to be understood in context, as broader forces such as
inequality, political division, and historical grievances can limit its
impact. Here, psychologists can fruitfully cooperate with other social
science disciplines.

Studying the most meaningful forms of contact -- deep, sustained
relationships forged over years -- poses particular challenges. Such
relationships cannot be randomly assigned, develop slowly, and are
difficult to measure without disrupting them. Creative, flexible designs
are therefore needed, with interpretations that acknowledge their
limitations. Current research focuses on making contact more effective
and lasting. Promising approaches include pairing it with
perspective‐taking, cooperative learning in schools, norm‐shaping media
campaigns, or virtual‐reality simulations of positive encounters. Some
initiatives also embed contact in long‐term community projects or
redesign institutions, such as integrated workplaces or mentoring
networks, so diverse cooperation becomes part of daily life. The
challenge ahead is to move from showing that contact can work to
understanding how to make it work consistently, at scale, and for the
long term.

\begin{quote}
{\#yourturn}

The IAT was designed to assess automatic associations people may hold
unconsciously. However, if implicit and explicit attitudes are highly
correlated, what are the implications for how we understand the
relationship between conscious and unconscious mental processes?
\end{quote}

\bookmarksetup{startatroot}

\chapter{\texorpdfstring{{Implicit Association Test and
Attitudes}}{Implicit Association Test and Attitudes}}\label{implicit-association-test-and-attitudes}

{written by Karolin Kessel (original draft), Bradley Baker (revision),
Savannah C. Lewis (revision)}

\section{1. The Classic}\label{the-classic-4}

How can we know what people truly think about a certain topic? This
question is difficult to answer, given that people are sometimes
motivated to misreport their true attitudes.

\begin{quote}
\phantomsection\label{def-attitude}{\#definition} Definition of
``attitude''

The cognition, affect and behavioral tendencies towards a certain
object.
\end{quote}

For example, if a friend was very excited about a new band they
discovered, you might feel like you don't want to burst their bubble of
joy by telling them you don't enjoy the music as much as your friend.
Because it is socially desirable to respond positively, to mirror your
friends' liking of the band, you might misreport your true attitude to
them.

\begin{quote}
{\#yourturn}

Think back to a time you thought or felt differently from what you
expressed publicly. Why did you not report the truth?
\end{quote}

Researchers in social psychology have been working on ways to assess and
measure people's attitudes towards a multitude of different topics. The
Implicit Association Test (IAT), developed by Greenwald, McGhee, and
Schwartz (\citeproc{ref-Greenwald-et-al_1998}{1998}), is a psychological
tool to measure implicit attitudes that people may not be aware of or
may not openly express.

\begin{quote}
\phantomsection\label{def-implicitattitude}{\#definition} Implicit
Attitude

An enduring mental disposition toward something that is not consciously
identified and of which a person may lack awareness.
\end{quote}

The test works by measuring how quickly people process and respond to
pairs of words or images. It relies on the idea that people respond
faster when two concepts that are closely linked -- or associated -- in
their mind (a congruent association) are paired than when the pairing
feels mismatched or unrelated (an incongruent association,
\citeproc{ref-Jhangiani-Tarry_2022}{Jhangiani and Tarry 2022}).

\begin{quote}
\phantomsection\label{def-association}{\#definition} Association

``A connection or relationship between two items (e.g., ideas, events,
feelings) with the result that experiencing the first item activates a
representation of the second'' (\citeproc{ref-APA-dict_2018}{{``{APA
Dictionary} of {Psychology}''} 2018}).
\end{quote}

\begin{quote}
\phantomsection\label{def-congruentassociation}{\#definition} Congruent
Association

A mental relationship between two objects or concepts characterized by
agreement or harmony.
\end{quote}

\begin{quote}
\phantomsection\label{def-incongruentassociation}{\#definition}
Incongruent Association

A mental relationship between two objects or concepts characterized by
lack of harmony or misalignment.
\end{quote}

Measuring these reaction times allows researchers to understand the
strength of automatic associations between concepts. Being able to
measure implicit attitudes provides a different perspective on how
people feel than is available from simply asking people to report their
attitudes, either because people may not realize their preferences or
may not be willing to share them. The IAT is designed to reveal
unconscious biases and remove bias that can be introduced by people
simply giving answers they think are socially acceptable, rather than
what they truly believe.

\begin{quote}
\phantomsection\label{def-implicitassociationtest}{\#definition}
Implicit Association Test

A reaction-time task that measures the strength of automatic
associations between concepts (e.g., flowers and positivity) by
comparing how quickly people classify paired categories. Faster
responses indicate stronger underlying associations.
\end{quote}

In the classic study by Greenwald, McGhee, and Schwartz
(\citeproc{ref-Greenwald-et-al_1998}{1998}), participants completed a
task called the Implicit Association Test (IAT) to measure automatic
associations. They were asked to quickly sort words and pictures into
four groups: two groups of objects (like flowers and insects) and two
groups of feelings (pleasant and unpleasant words). The test had two
main parts. In one part, participants pressed the same key for flowers
and pleasant words, and another key for insects and unpleasant words. In
the other part, the pairings were switched: flowers with unpleasant
words and insects with pleasant words. Participants had to respond as
fast and accurately as possible. If someone naturally associates flowers
with pleasantness, they will respond faster when those two categories
are paired together.

\begin{quote}
{\#yourturn}

Why would researchers not simply ask participants about their attitudes?
\end{quote}

This approach is used to capture unconscious connections between
concepts in memory, which in the original test format were aimed at
assessing implicit stereotypes and prejudices, but have been used to
identify a variety of subtle attitudes in various subject areas
(\citeproc{ref-Nosek-Smyth_2007}{Nosek and Smyth 2007}).

\begin{quote}
{\#yourturn}

Do you think that reaction times or spontaneous reactions are an
appropriate measure of implicit cognitions such as stereotypes? Why or
why not?
\end{quote}

\section{2. The Aftermath}\label{the-aftermath-4}

In his study, ``The Implicit Association Test: A Method in Search of a
Construct,'' Ulrich Schimmack (\citeproc{ref-Schimmack_2021}{2021})
examines the power of the IAT in revealing individual differences in
implicit social cognition.

\begin{quote}
\phantomsection\label{def-implicitsocialcognition}{\#definition}
Implicit Social Cognition

The automatic, unconscious mental processes that influence how we
perceive, evaluate, and interact with others.
\end{quote}

The results show that there is insufficient evidence for the construct
validity of the test (\citeproc{ref-Schimmack_2021}{Schimmack 2021}), in
other words, that there is not enough proof that the IAT measures what
it was intended to measure (implicit bias), rather than something else.
This can be seen when scores from a test aren't related to other
measures of the same concept in the expected ways. Based on examination
of several multimethod studies, Schimmack found little or no evidence of
discriminant validity compared to measures of explicit attitudes, making
it unclear whether the test really captures a different type of attitude
(implicit, rather than explicit). Problems with discriminant validity
show up when a measure's scores are too similar to those of an
established measure for a different concept, making it unclear whether
the new measure is assessing something unique. That is, Schimmack
(\citeproc{ref-Schimmack_2021}{2021}) raises questions regarding a lack
of evidence that the IAT adequately measures individual-level
differences in implicit associations and the extent to which the IAT
measures something different from self-report measures of explicit
associations.

\begin{quote}
\phantomsection\label{def-multimethodstudy}{\#definition} Multimethod
Study

Research that employs two or more distinct methods.
\end{quote}

\begin{quote}
\phantomsection\label{def-discriminantvalidity}{\#definition}
Discriminant Validity

The extent to which a test is unrelated to measures designed to assess
theoretically distinct constructs.
\end{quote}

\begin{quote}
\phantomsection\label{def-constructvalidity}{\#definition} Construct
Validity

The extent to which a test measures the theoretical construct or concept
it is intended to measure.
\end{quote}

Schimmack highlights that these deficiencies have been overlooked for
many years and finds that explicit measures are more valid than the IAT
in all areas. This means simply asking participants about their
attitudes might indeed be the better measure of these attitudes than
making them take the IAT. At the same time, Schimmack also argues that
the IAT can be used as a complementary measurement tool to explicit
measures for sensitive settings to reduce measurement errors by
employing a multi-method measurement model. In other words, using both
explicit measures and the IAT might be the best approach.

\begin{quote}
{\#yourturn}

What are the pros and cons of using this kind of test in bias training?
\end{quote}

\section{3. Conclusion}\label{conclusion-4}

The establishment of implicit association testing resulted in one of the
most influential articles in personality and social psychology
(\citeproc{ref-Greenwald-et-al_1998}{Greenwald, McGhee, and Schwartz
1998}), and established the foundation for a variety of new (social
psychological) theories (\citeproc{ref-Schimmack_2021}{Schimmack 2021}).
At the same time, the difficulties identified by Schimmack illustrate
the extent to which social psychological theory formation is highly
complex. Particularly when investigating the discrepancy between human
thinking and socially desirable conformity, as well as its (uncertain)
influence on behavior, precise (construct) differentiation and validity
testing are essential in research. Extensive research has used and built
on the IAT (\citeproc{ref-Greenwald-et-al_2003}{Greenwald, Nosek, and
Banaji 2003}; \citeproc{ref-Greenwald-et-al_2009}{Greenwald et al.
2009}) and related approaches to measuring the strength of automatic
associations or using implicit measures to bypass bias in research data
due to socially desirable responding by study participants.

\begin{quote}
\phantomsection\label{def-socialdesirability}{\#definition} Social
Desirability

The tendency to want to be viewed positively by others, often by
aligning with socially approved behaviors and attitudes.
\end{quote}

\begin{quote}
\phantomsection\label{def-sociallydesirableresponding}{\#definition}
Socially Desirable Responding

The act of providing inauthentic responses to better present oneself
favorably according to current social norms.
\end{quote}

However, Schimmack (\citeproc{ref-Schimmack_2021}{2021}) highlighted
weaknesses of the IAT regarding construct and discriminant validity as a
measure of implicit constructs. He emphasizes the significance of being
cautious when making claims about subtle ideas based on the IAT and
highlights the variation in the IAT's validity depending on the
construct being measured. If the IAT is not measuring implicit attitudes
(unconscious biases) or does not provide additional information beyond
simply asking people about their attitudes, as suggested by Schimmack,
then the test offers limited utility to researchers and calls into
question findings that rely on the IAT.

\begin{quote}
{\#yourturn}

The IAT was designed to assess automatic associations people may hold
unconsciously. However, if implicit and explicit attitudes are highly
correlated, what are the implications for how we understand the
relationship between conscious and unconscious mental processes?
\end{quote}

\bookmarksetup{startatroot}

\chapter{\texorpdfstring{{False Consensus
Effect}}{False Consensus Effect}}\label{false-consensus-effect}

{written by Marcel Zubrod (original draft), Jana Berkessel (revision),
and Márton Kolozsvári (revision)}

\section{The Classic}\label{the-classic-5}

The false consensus effect is a cognitive bias in which individuals
overestimate the extent to which their own beliefs, preferences, and
behaviors are shared by others.

\begin{quote}
\phantomsection\label{def-bias}{\#definition} Bias

A systematic distortion of perception or judgment.
\end{quote}

This psychological phenomenon was first systematically studied by Ross,
Greene, and House (\citeproc{ref-ross_false_1977}{1977}), who
demonstrated that individuals tend to perceive their own choices and
opinions as more common than they actually are. For instance, people who
express a preference for a particular option are likely to assume that
others would make the same choice, even when evidence suggests
otherwise. This bias occurs because individuals use their own
perspective as a reference point, leading to distorted judgments about
the preferences, opinions and behaviors of others.

\begin{quote}
\phantomsection\label{def-falseconsensus}{\#definition} False Consensus
Effect

A cognitive bias where individuals overestimate the extent to which
others share their beliefs, preferences, and behaviors.
\end{quote}

In Study 1 of the original research by Ross, Greene, and House
(\citeproc{ref-ross_false_1977}{1977}), participants were presented with
one of four short stories, each describing a fictional scenario with a
behavioral choice to be made. After reading the assigned story,
participants were asked to estimate the percentage of their peers who
would choose one behavioral option over the other within the context of
the story.

\begin{quote}
{\#yourturn}

Can you think of a time when you assumed others thought or behaved the
same way you did and it turned out to not be the case?
\end{quote}

Following these percentage estimates, participants completed a
questionnaire. First, they were required to indicate which behavioral
option they personally would have chosen in the scenario. Next, they
rated themselves on a personality scale. As part of the assessment,
participants also evaluated the typical personality characteristics of
someone their age and gender who would choose either behavioral option
presented in the story.

The results revealed a consistent pattern: participants who chose a
particular behavioral option tended to believe that ``people in
general'' would likely make the same choice. Conversely, participants
who rejected an option perceived that behavior as less likely for
others. Across all four stories, participants' own choices strongly
predicted their estimates of how the general population would behave.

Additionally, significant differences emerged in personality evaluations
based on participants' own choices. For three of the four stories,
participants rated the typical personality traits of those choosing
their preferred behavioral option as less extreme than those who
selected the alternative. These effects were statistically significant
in three stories, while one story showed a weaker significance, and the
fourth story showed no significant results.

\begin{quote}
{\#yourturn}

Are there certain methodological choices that could enhance or reduce
the magnitude of the false consensus effect? These could include, but
are not limited to, the number of choices to choose from, the social
setting, the controversiality of the choices and the order of choices.
Do they increase or reduce the magnitude of the false consensus effect?
\end{quote}

\section{The Aftermath}\label{the-aftermath-5}

A meta-analysis by Mullen et al.
(\citeproc{ref-mullen_false_1985}{1985}) examined 23 studies and a total
of 115 hypotheses related to the false consensus effect. The analysis
demonstrated that tests for the false consensus effect were highly
significant and produced a moderate effect size. Importantly, it
identified specific methodological factors that influenced the magnitude
of the effect. For instance, the number of behavioral decisions
participants were asked to make, as well as the order in which decisions
and consensus estimates were presented, significantly impacted the
observed false consensus effect.

\begin{quote}
\phantomsection\label{def-effectsize}{\#definition} Effect Size

A quantitative measure of the magnitude of a phenomenon, used to assess
the practical significance of research findings.
\end{quote}

These findings suggested that subsequent studies should limit the number
of behavioral decisions participants are required to make and prioritize
consensus assessments before behavioral decisions, as those
methodological peculiarities might maximize the observed extent of the
false consensus effect in experimental settings.

The \textbf{self-presentation explanation} posits that individuals
strategically align their behavior with perceived social norms.
According to this theory, the false consensus effect should be more
pronounced when individuals make their behavioral decision before
estimating the consensus. Only in this sequence do participants have the
chance to adjust the social norm (i.e., other people's behavior) to
their own behavior. However, the meta-analysis by Mullen et al.
(\citeproc{ref-mullen_false_1985}{1985}) found no statistical evidence
supporting this prediction, suggesting that the false consensus effect
does not vary as the self-presentation explanation would anticipate.

\begin{quote}
{\#yourturn}

Which other mechanisms could explain the False Consensus Effect? How
would you test those mechanisms?
\end{quote}

mullen\_false\_1985 outlined several theoretical explanations for the
false consensus effect. One explanation, \textbf{attributive
projection}, suggests that individuals rely on cognitive biases to
justify their belief that their own behavioral choices are rational and
appropriate responses to the environment. Another perspective suggests
that the false consensus effect can \textbf{protect a person's
self-esteem}. It may help people feel better about themselves when they
face failure or receive negative feedback about their personal
characteristics. A third explanation focuses on \textbf{social
environments}, noting that people tend to associate with others who
share similar backgrounds, values, and interests. Using false consensus
makes us associate with the others who are (often falsely) perceived to
be similar, thus fulfilling the need for a sense of relatedness. This
selective association reinforces the perception that their choices are
widely shared. Finally, \textbf{cognitive availability} provides a more
mechanistic account, proposing that the behaviors individuals have
chosen---or would choose---are more easily recalled or imagined than
alternative actions when theorising about the behavior of others, a
phenomenon linked to the availability heuristic.

\begin{quote}
\phantomsection\label{def-availabilityheuristic}{\#definition}
Availability Heuristic

A mental shortcut where people estimate the likelihood of an event based
on how easily examples come to mind, which can lead to overestimating
rare but memorable occurrences.
\end{quote}

Overall, the false consensus effect is often attributed to a
psychological desire to see one's thoughts and actions as appropriate,
normal, and correct. Together, these cognitive and motivational factors
help explain why individuals consistently overestimate the prevalence of
their own opinions and behaviors, a phenomenon observed across numerous
studies.

Recent research has refined our understanding of the false consensus
effect, particularly by situating it in contemporary social and digital
contexts. In a series of studies, Bunker and Varnum
(\citeproc{ref-bunker_how_2021}{2021}) found that greater social media
use was reliably associated with stronger false consensus effects across
domains such as political attitudes, personality traits, and social
motives. However, the size of these effects was consistently smaller
than laypeople anticipated, suggesting a public overestimation of social
media's distorting power. Luzsa and Mayr
(\citeproc{ref-luzsa_false_2021}{2021}) experimentally demonstrated that
exposure to attitudinally congruent news feeds, especially those with
high agreement and visible endorsement cues like ``likes'', leads
individuals to overestimate public support for their own views.
Interestingly, this inference was moderated by participants' interest in
the topic, with highly engaged individuals showing more skepticism
toward consensus cues.

Building on the political implications of false consensus, Steiner,
Landwehr, and Harms (\citeproc{ref-steiner_false_2025}{2025}) found that
individuals who overestimate how many others share their political
preferences are more likely to express populist attitudes and to
distrust political elites. Similarly, Weinschenk, Panagopoulos, and
Linden (\citeproc{ref-weinschenk_democratic_2021}{2021}) showed that
individuals' views about democratic norms, such as the peaceful transfer
of power, were strongly linked to their perceptions of what others
believe---indicating a false consensus bias, particularly among
conservatives. Finally, Furnas and LaPira
(\citeproc{ref-furnas_people_2024}{2024}) extended the scope of the
false consensus effect to unelected political elites (e.g., lobbyists
and journalists) demonstrating that this group's perceptions of public
opinion systematically reflected their own views, suggesting egocentrism
rather than ideological bias as the driving force.

Together, these studies demonstrate that the false consensus effect is a
robust phenomenon with wide-ranging relevance from digital communication
to political judgment and that it is shaped not only by cognitive
mechanisms but also by the structural, technological, and ideological
environments in which opinions are formed.

\section{Conclusion}\label{conclusion-5}

The body of research on the false consensus effect highlights its
robustness as a psychological phenomenon while also revealing important
complexities in how it comes about. Early experimental studies, such as
those by Ross, Greene, and House (\citeproc{ref-ross_false_1977}{1977}),
demonstrated that individuals consistently overestimate the degree to
which others share their beliefs and behaviors. Follow-up meta-analyses,
like that of Mullen et al. (\citeproc{ref-mullen_false_1985}{1985}),
confirmed the effect's significance and explored the methodological and
contextual factors that influence its magnitude.

In the broader context of social psychology, the false consensus effect
provides valuable insights into how cognitive biases and motivational
factors shape human perception. Explanations for the effect, ranging
from attributive projection and ego defense to mechanisms like cognitive
availability, underline the interplay between how individuals view
themselves and how they perceive the social world around them.

However, as with many constructs in psychology, it is crucial to
approach findings on the false consensus effect with careful scrutiny.
Methodological variations can significantly impact the observed
magnitude of the effect, and further research is needed to disentangle
its underlying mechanisms. The enduring study of the false consensus
effect is an example of the importance of revisiting and refining
theoretical constructs to build a more comprehensive understanding of
human cognition and behavior.

\bookmarksetup{startatroot}

\chapter{\texorpdfstring{{Facial Feedback
Hypothesis}}{Facial Feedback Hypothesis}}\label{facial-feedback-hypothesis}

{written by Sophia Reitmayer (original draft), Patrícia Arriaga
(revision), and Effy Zachou (revision).}

\section{The Classic}\label{the-classic-6}

Does what your body does influence how you feel? This is a central
question that the Facial Feedback hypothesis addresses. The idea is
simple, and quite old. In fact, it echoes one of the earliest theories
of emotions in modern psychology: the James-Lange theory of emotion
(\citeproc{ref-james_what_1884}{James 1884}). This theory proposes that
bodily changes precede and give rise to emotional experiences. In other
words, perhaps what our body does informs what we feel.

\begin{quote}
{\#yourturn}

Have you ever felt your heart beat faster when giving a presentation or
walking into a room full of people, and then noticed yourself feeling
nervous or fearful? These are examples of how bodily responses, such as
a racing heart or sweating, might shape emotional experience, as
suggested by James-Lange theory (\citeproc{ref-james_what_1884}{James
1884}).
\end{quote}

Now think more specifically: have you ever noticed that frowning while
concentrating made you feel more tense? Or that you felt more positive
when you smiled, even without a clear reason? These are everyday
examples of how facial expressions, as specific bodily reactions, might
affect your emotional state, as proposed by the Facial Feedback
Hypothesis.

The Facial Feedback Hypothesis can also be related to the work of Darwin
(\citeproc{ref-darwin_expression_1872}{1872}) and, later, Ekman
(\citeproc{ref-ekman_argument_1992}{1992}), as both suggested that
facial expressions play a role in emotion. Ekman, for example,
emphasized that certain facial expressions are universal and
biologically innate. However, these theories are distinct, since unlike
the Facial Feedback Hypothesis, neither Darwin nor Ekman proposed that
facial expressions causally influence the emotional experience itself.
In contrast, the Facial Feedback Hypothesis suggests that the activation
of facial muscles involved in an expression can modulate the subjective
experience of emotion. This theory posits that the act of forming a
facial expression, such as smiling, frowning, or furrowing the brow, can
intensify, initiate, or modulate the corresponding emotional state,
thereby establishing a bidirectional relationship between expression and
affect. Thus, the act of smiling may actually make people feel happier.

\begin{quote}
{\#yourturn}

Can you think of everyday situations where the Facial Feedback
Hypothesis might apply? Try to go beyond smiling, by considering how
other facial expressions might also shape your emotional experience,
such as sadness, anger, fear, disgust.
\end{quote}

The publication by Strack, Martin, and Stepper
(\citeproc{ref-strack_inhibiting_1988}{1988}) investigated this
hypothesis in two studies. The authors tested whether adopting a facial
expression typically associated with a specific emotion could influence
people's emotional experience and their evaluation of external stimuli.
More specifically, they investigated whether producing a smiling facial
expression could lead to a more positive evaluation of cartoons and a
more positive emotional state.

Strack, Martin, and Stepper
(\citeproc{ref-strack_inhibiting_1988}{1988}) conducted two studies
using a new methodology designed to prevent a cognitive interpretation
of facial action. In other words, the aim was to avoid participants
consciously recognising their facial movements as expressions of
specific emotions. This was important because one of the main concerns
is the risk of demand characteristics, that is, the possibility that
participants' awareness of the study's true purpose might influence
their responses. To address this, they introduced a cover story, telling
participants that the study focused on psychomotor coordination. This
procedure became known as the ``pen-in-the-mouth'' paradigm, allowing
for a more subtle manipulation of facial muscle activity.

In both studies, participants (N = 92, Study 1; N = 83, Study 2) used
the same pen-in-the-mouth paradigm. In study 1, participants were
assigned to three conditions. In one condition, participants were asked
to hold a pen between their teeth in a way that would facilitate a
facial configuration associated with smiling (``teeth'' condition). In
this condition, the way participants held the pen would activate the
facial zygomaticus major muscles, which are typically involved in
smiling.

\begin{quote}
\phantomsection\label{def-zygomaticusmajor}{\#definition} Zygomaticus
Major Muscles

These bilateral facial muscles, when activated, raise the corners of the
mouth in an upward and lateral direction, facilitating expressions such
as smiling.
\end{quote}

In a second condition, they were asked to hold the pen between their
pursed lips (``lips'' condition). In contrast to the ``teeth''
condition, this position engages the orbicularis oris muscles, which may
inhibit the activation of the zygomaticus major, making smiling more
difficult.

\begin{quote}
\phantomsection\label{def-orbicularisoris}{\#definition} Orbicularis
Oris Muscles

These are circular muscles around the mouth that close the lips and
produce puckering, as in kissing or whistling.
\end{quote}

The third condition included in study 1 served as a control group, as it
did not involve any direct manipulation of the facial muscles. Instead,
participants were asked to hold the pen with their non-dominant hand.

In study 1, the aim was to test whether facial manipulation influenced
the evaluation of humorous stimuli (perceived funniness) and study 2
aimed at replicating the procedure but also differentiating the effects
on cognitive and affective components of this response. Thus, after
being assigned to one of the conditions, participants were presented
with cartoons on various topics, ranging from neutral to humorous
situations, and asked to rate how funny each cartoon was on a scale of 0
to 9 (``not at all funny'' to ``very funny''). Additionally, in study 2,
the affective experience of amusement was measured by asking
participants to indicate how amused they felt while viewing the
cartoons, also using a 10-point scale (from 0 = ``I felt not at all
amused'' to 9 = ``I felt very much amused'').

The results in study 1 showed differences in the ratings of the cartoons
between the ``teeth'' and ``lips'' conditions. In the ``teeth''
condition, participants rated the cartoons as significantly funnier than
in the ``lips'' condition, and the results of the control group fell
between these two conditions. This suggests that activating the facial
muscles involved in smiling can lead to a more positive perception of
humorous stimuli, while inhibiting those muscles reduces this positive
perception. In study 2, by introducing separate measures for cognitive
and affective components, the authors showed that facial manipulation
affected only the amusement experience without affecting the cognitive
evaluation of funniness. This highlights the need to distinguish between
these two components explicitly. According to the authors, the effects
obtained in the perceived funniness of the cartoons in study 1 likely
reflected a combination of affective and cognitive influences within a
single global evaluation.

\begin{quote}
{\#yourturn}

Why do you think Strack et al.'s (1988) publication was so influential?
Are you fully convinced? Are there exceptions to the rule? How could
facial-feedback be criticized?
\end{quote}

Over the years, several questions have been raised, and both conceptual
and direct replications of Strack et al.'s
(\citeproc{ref-strack_inhibiting_1988}{1988}) study have been conducted.
For example: Are facial feedback effects stronger when people produce
genuine, spontaneous smiles, compared to subtle and artificial
manipulations like holding a pen in the mouth? Does facial feedback
initiate emotional experiences, or does it merely amplify emotions that
are already present? Also, although Strack et al.
(\citeproc{ref-strack_inhibiting_1988}{1988}) focused specifically on
smiling, the Facial Feedback Hypothesis suggests that other facial
expressions may also contribute to shaping emotional experience.

\section{The Aftermath}\label{the-aftermath-6}

Strack et al.'s (\citeproc{ref-strack_inhibiting_1988}{1988})
influential study has been the subject of debate in recent years, as
several researchers have had difficulties replicating the original
results. One of the attempts was the Registered Replication Report (RRR)
by Wagenmakers et al.
(\citeproc{ref-wagenmakers_registered_2016}{2016}). Despite coordination
across 17 independent laboratories, the replication failed to reproduce
original findings: participants did not rate cartoons as funnier when
their facial muscles were configured into a smile. This null result
raised doubts about how reliable the facial feedback hypothesis is. In
response, Strack (\citeproc{ref-strack_reflection_2016}{2016}) argued
that small differences in the setting, especially the use of video
cameras, may have affected the participants' responses. He suggested
that being watched could make people more self-aware and stop the
natural reactions needed for facial feedback to work. Later, Noah,
Schul, and Mayo (\citeproc{ref-noah_when_2018}{2018}) investigated this
concern by examining whether the presence of video cameras could alter
participants' behavior. In two experiments, they compared conditions
with and without video monitoring. The results showed that the
pen-in-mouth task influenced results only when participants were not
being observed. This suggests that the facial feedback effect is
influenced by whether people feel they are being monitored, and that
subtle changes in the study design can affect the results.

More recently, Coles et al. (\citeproc{ref-coles_multi-lab_2022}{2022})
contributed to this debate with the Many Smiles Collaboration, designed
as a large-scale, pre-registered multi-lab project to test the facial
feedback hypothesis through both direct and conceptual replications.

\begin{quote}
\phantomsection\label{def-conceptualreplication}{\#definition}
Conceptual Replication

A study that aims to recreate the gist of a prior study without using an
identical procedure. These studies often aim to explore boundary
conditions, the influence of specific variables, or aim to broaden and
extend a certain finding.
\end{quote}

Conducted across 19 countries with data from 3,878 participants, their
study used various methods to examine the reliability of facial feedback
effects. Participants were asked to imitate prototypical or less
prototypical facial expressions of happiness (facial mimicry paradigm)
or to perform voluntary facial movements (voluntary facial action). In
addition, the pen-in-the-mouth paradigm from Strack, Martin, and Stepper
(\citeproc{ref-strack_inhibiting_1988}{1988}) was used, in which
participants held the pen either between their teeth or between their
lips.

The results showed that, when present, the effects were small,
supporting the idea that facial feedback contributes to emotion but is
not its primary determinant. There was consistent evidence of emotional
amplification in voluntary smiling and mimicry tasks, while results for
the pen-in-mouth task were less clear, even when avoiding video
recording. It is worth noting that Strack was directly involved in this
project, highlighting the project's collaborative effort to test the
facial feedback hypothesis. The results suggest that different
mechanisms may underlie the effects of each task. Rather than refuting
the facial feedback hypothesis, Coles et al.'s
(\citeproc{ref-coles_multi-lab_2022}{2022}) findings frame it as a
conditional and modest phenomenon, dependent on how facial expressions
are elicited and on contextual factors such as participant awareness.

\section{Conclusion}\label{conclusion-6}

Attempts to replicate Strack et al.'s
(\citeproc{ref-strack_inhibiting_1988}{1988}) original findings have
produced inconsistent results. Importantly, the Many Smiles
Collaboration (\citeproc{ref-coles_multi-lab_2022}{Coles et al. 2022})
did not provide clear evidence regarding the emotional amplification
effect of the pen-in-mouth task used in Strack et al.'s
(\citeproc{ref-strack_inhibiting_1988}{1988}) study. However, this
recent project broadened the scope of investigation by including
additional paradigms, such as voluntary smiling and facial mimicry,
which yielded small but consistent facial feedback effects.

Overall, the evidence suggests that facial feedback can influence
emotional experience, but its effects are small, sensitive to context,
and not consistent across all types of manipulations. These studies also
highlight the importance of identifying the conditions under which
facial feedback operates.

From the perspective of James-Lange theory, the findings remain
consistent with the idea that bodily changes contribute to affective
experience, though in a more limited and conditional way than originally
assumed.

\begin{quote}
{\#yourturn}

In light of these results, would you say that smiling more will make
people feel happier?
\end{quote}

In short, the relationship between facial expressions and emotions is
complex. Such effects may occur, but they are usually small,
context-dependent, and further research is still needed to determine
when and how they emerge. Smiling alone is unlikely to serve as a simple
route to happiness.

\bookmarksetup{startatroot}

\chapter{\texorpdfstring{{Heat Priming-Hostile Perception
Effect}}{Heat Priming-Hostile Perception Effect}}\label{heat-priming-hostile-perception-effect}

{written by Hannes Dieterle (original draft), and Patrícia Arriaga
(revision)}

\section{1. The Classic}\label{the-classic-7}

Does what you have recently seen, heard or read affect how you think,
even if you do not realise it? This is a central question behind the
concept of priming, which has been used to describe how subtle cues,
like words related to temperature, might influence our thoughts,
feelings, and behaviours.

The verb ``to prime'' means ``to activate''. In psychology, ``priming''
refers to the idea that exposure to a stimulus can activate mental
representations, making it easier or faster to respond to that same
stimulus later (direct priming), or to something related to it (indirect
priming).

\begin{quote}
\phantomsection\label{def-priming}{\#definition} Definition of
``priming''

``Priming refers to facilitative effects of an encounter with a stimulus
on subsequent processing of the same stimulus (direct priming) or a
related stimulus (indirect priming)''
(\citeproc{ref-tulving_priming_1982}{Tulving, Schacter, and Stark 1982,
pg.336}).
\end{quote}

To test how such subtle verbal cues might affect person perception,
DeWall and Bushman (\citeproc{ref-nathan_dewall_hot_2009}{2009})
conducted an experiment (Study 2) in which they investigated the
relationship between exposure to words associated with hot and cold
temperatures and the subsequent evaluation of a fictitious person. The
72 undergraduate students who participated in this experiment were first
randomly assigned to one of three groups, in which they were primed with
temperature-related or neutral words.

Their task consisted of creating grammatically correct sentences from
five scrambled words. In the ``heat prime'' and ``cold-prime'' groups,
six of the 13 sentences contained words associated with heat or cold,
respectively. The ``neutral prime'' group's task did not include any
temperature-associated words; therefore, it served as the ``control
group''.

A ``control'' group is often used as a baseline in experiments, allowing
researchers to see whether the changes observed in the experimental
groups are due to the manipulation, and not to other factors. In this
study, the control group was created to test whether exposure to ``hot''
or ``cold'' words influenced how participants judged the fictitious
person, compared to a group with no temperature cues.

Thus, the priming condition, with the three levels (heat, cold, and
neutral), was the independent variable (IV) in this experiment.
Subsequently, all participants read a text about a fictitious man named
Donald, whose behaviour was described in an ambiguous but potentially
hostile manner. Participants were asked to rate Donald's personality in
four questions related to hostility traits. The responses to these four
questions were combined into an index of hostile perception, which
served as the dependent variable (DV).

\begin{quote}
{\#yourturn}

Why did the researchers measure the perception of Donald's personality
after participants were primed with the concepts of heat or cold,
compared to the neutral control group?
\end{quote}

The underlying assumption is that priming can increase the accessibility
of specific personality-related concepts or trait descriptions in
memory, which in turn may shape how ambiguous information about others
is interpreted (\citeproc{ref-srull_role_1979}{Srull and Wyer 1979}).
Additionally, theoretical models such as the General Aggression Model
(\citeproc{ref-anderson_temperature_nodate}{Anderson and Anderson 1998})
integrate the temperature--aggression hypothesis, proposing that hot
temperatures can serve as situational inputs that activate
aggression-related thoughts and feelings. In Study 2, DeWall and Bushman
(\citeproc{ref-nathan_dewall_hot_2009}{2009}) tested the more specific
hypothesis that exposure to heat-related words would increase hostile
perceptions of an ambiguously described person, compared to both neutral
and cold-related words.

To compare the groups, the authors adopted the null hypothesis
significance testing approach (NHST,
\citeproc{ref-brandt_replication_2014}{Brandt et al. 2014};
\citeproc{ref-cumming_new_2014}{Cumming 2014};
\citeproc{ref-wasserstein_asa_2016}{Wasserstein and Lazar 2016}), by
comparing the mean scores on the hostility index across the three
priming conditions. In this approach, a result is considered
statistically significant when the probability of observing a difference
is sufficiently low, typically less than 5\% (p \textless{} .05),
assuming that there is actually no real difference between groups (the
null hypothesis). As is typical in psychological research, the authors
used this threshold to determine whether the differences between group
means were statistically significant.

DeWall and Bushman (\citeproc{ref-nathan_dewall_hot_2009}{2009}) results
showed that the ``heat prime'' group rated Donald as significantly more
hostile than both the ``neutral'' and the ``cold'' groups (heat
vs.~cold: d = .67, p \textless{} .03; heat vs.~neutral: d = .63, p
\textless{} .05). Moreover, no significant differences were found
between the ``cold'' and the ``neutral'' groups (p = .85).

These findings suggest that exposure to heat-related words increased
participants' tendency to perceive ambiguous behavior as more hostile,
supporting the hypothesis that temperature-related concepts can activate
hostility-related trait perception. The absence of a statistical
difference between the ``cold'' and ``neutral'' groups (p \textgreater{}
.05) further indicates that this effect was specific to heat-related
priming, rather than a general effect of temperature-related concepts.

\section{2. The Aftermath}\label{the-aftermath-7}

In 2014, McCarthy conducted two replication studies of this experiment
(\citeproc{ref-mccarthy_close_2014}{McCarthy 2014}).

\begin{quote}
{\#yourturn}

What criteria should a replication meet in order to be relevant and
helpful for examining the effect?
\end{quote}

There are different types of replication studies, each with different
criteria and goals, although both aim to test the same theoretical
claims. Close replications aim to verify whether the original effect can
be found again under the same conditions as the original study, using
the same method. In contrast, conceptual replications test the
generalisability of an effect across contexts and may rely on different
operational definitions or use a different method
(\citeproc{ref-brandt_replication_2014}{Brandt et al. 2014}).

By following Brandt et al.'s
(\citeproc{ref-brandt_replication_2014}{2014}) definition and guidelines
for close replication, McCarthy
(\citeproc{ref-mccarthy_close_2014}{2014}) designed two studies aiming
to reproduce the original procedures but using larger samples. He
justified these attempts to replicate with three arguments. First, a
single study is not sufficient to establish the reliability of an
effect, and further testing is necessary. Second, the original study had
a relatively small sample size, which can lead to unstable effect size
estimates; and larger samples are required. Third, the original findings
had already been widely cited, so it is important to verify whether they
could be replicated before treating them as reliable knowledge.

In McCarthy's (\citeproc{ref-mccarthy_close_2014}{2014}) first
replication study, involving 182 participants, participants were
randomly assigned to one of three priming conditions (heat, cold, or
neutral) and completed the same scrambled sentence task as in the
original. After the priming task, they read the same ambiguous story
about a fictitious man and rated his hostility using the same four items
to measure hostile perceptions. The second replication, conducted online
with 507 participants, used the same critical heat- and cold-related
words as in the original experiment, but the scrambled sentences in
which these words appeared were slightly different from the original
materials. Otherwise, the procedure closely followed the original study.

The results of these two replication studies did not support the
original hypotheses. Donald's rated hostility did not differ
significantly between the heat and the cold prime groups. Thus, the
findings reported by DeWall and Bushman
(\citeproc{ref-nathan_dewall_hot_2009}{2009}) could not be replicated.
Additionally, McCarthy (\citeproc{ref-mccarthy_close_2014}{2014})
conducted a meta-analysis combining the original study with the two
replications. This analysis also indicated a non-significant effect of
heat priming on hostile perceptions (d = 0.18, p \textless{} .05). Based
on these results, McCarthy (\citeproc{ref-mccarthy_close_2014}{2014})
concluded that priming individuals with heat-related concepts does not
reliably affect hostile perceptions of others, and that the original
effect is likely non-existent or too weak to be considered meaningful.

\section{3. Conclusion}\label{conclusion-7}

McCarthy (\citeproc{ref-mccarthy_close_2014}{2014}) tried to replicate
DeWall and Bushman's (\citeproc{ref-nathan_dewall_hot_2009}{2009}) study
twice and found no evidence that heat-related words increase hostile
perceptions. Their meta-analysis combining the original and replication
studies also showed a non-significant effect, suggesting the original
finding is likely non-existent or too weak to be relevant.

This replication failure reflects a broader debate in psychology that
social priming effects may be fragile and difficult to reproduce,
particularly when it comes to temperature-related words and their
relation to hostility. Moreover, these studies evaluated hostile
perceptions rather than aggressive behaviour. Therefore, the findings do
not directly challenge broader theoretical models such as the General
Aggression Model (\citeproc{ref-anderson_temperature_nodate}{Anderson
and Anderson 1998}), which integrates the temperature--aggression
hypothesis through a variety of situational, cognitive, and affective
mechanisms. What the replications do suggest is that simple word-based
priming of hot and cold temperature is unlikely to be a reliable
predictor of person perception in terms of hostility.

\bookmarksetup{startatroot}

\chapter{\texorpdfstring{{Hot Coffee
Effect}}{Hot Coffee Effect}}\label{hot-coffee-effect}

{written by Aslı Ay Arat (original draft), and Aswathi Surendran
(revision)}

\section{The Classic}\label{the-classic-8}

Our environment often exerts strong influences on us. For instance,
seasonal changes in sunlight hours can seriously affect mood and
wellbeing (e.g., seasonal affective disorder), and people tend to be
more willing to make donations in December (also referred to as the
Christmas effect).

\begin{quote}
{\#yourturn}

How does your current environment affect you? Take a moment to reflect!
\end{quote}

Much psychological research is interested in how environments shape
human behavior, our thinking and what we feel (often also considered in
interaction with person-specific variables). One subfield has dedicated
research on embodied cognition -- the idea that bodily states influence
what and how we think and feel (\citeproc{ref-chabris_no_2019}{Chabris
et al. 2019}). The assumption is that the environment acts on the mind,
via the body.

\begin{quote}
{\#yourturn}

Based on the principles of embodied cognition, how might working in a
cluttered or messy room affect someone's ability to concentrate or
study?
\end{quote}

In a 2008 study, researchers Williams and Bargh
(\citeproc{ref-williams_experiencing_2008}{2008}) worked on a related
question. They wanted to know if temperature -- a salient feature of the
environment if you think about how often conversations are centered on
the weather -- affected how people are perceived. They hypothesized that
``physical warmth should activate concepts or feelings of interpersonal
warmth'' (\citeproc{ref-williams_experiencing_2008}{Williams and Bargh
2008}, pg. 3).

\begin{quote}
\phantomsection\label{def-psychologicalwarmth}{\#definition} Definition
of psychological warmth

The sense that another person is friendly, kind, and has good
intentions. It is one of the two central dimensions we rely on when
forming first impressions, the other being competence. People tend to
notice warmth quickly and often use it as a basis for deciding whether
someone is trustworthy. Importantly, research suggests that experiences
of physical warmth can subtly shape these social judgements
(\citeproc{ref-williams_experiencing_2008}{Williams and Bargh 2008};
\citeproc{ref-fiske_universal_2007}{Fiske, Cuddy, and Glick 2007}).
\end{quote}

The researchers asked undergraduate subjects to hold either a warm cup
of coffee or iced coffee in their hands while writing down information.
The expectation was that the concepts of warmth (or coldness) would be
primed due to the physical experience of the temperature of the coffee,
making it more likely that a person was correspondingly perceived as
warm (or cold).

\begin{quote}
\phantomsection\label{def-priming}{\#definition} Definition of
``priming''

``A change in how easily we recognise or produce something because of an
earlier encounter with it. In other words, our previous experience with
an item can make us faster or more accurate in responding to it later,
even if we are not consciously aware of the connection'' Tulving,
Schacter, and Stark (\citeproc{ref-tulving_priming_1982}{1982}).
\end{quote}

In the first part of the study, participants were undergraduate students
at Yale University. They were asked to hold either a warm cup of coffee
or an iced coffee while evaluating a fictional individual described in a
personality profile. Those who held the warm beverage rated the
individual as significantly more ``interpersonally warm'' compared to
those who held the cold beverage
(\citeproc{ref-williams_experiencing_2008}{Williams and Bargh 2008}).
This result was interpreted as evidence that the feeling of physical
warmth can unconsciously bring to mind the idea of social warmth. In
other words, holding something warm made people more likely to see the
person in the profile as kind and friendly.

In a follow-up study, participants were asked to hold either a heated or
a cold therapeutic pad, under the impression that they were simply
evaluating the product. Afterwards, they were given a choice of reward
for taking part in the study. They could either select a gift for
themselves (such as a drink voucher or an ice cream certificate) or
choose the same type of gift for a friend. The results showed a clear
pattern. Those who had held the warm pad were more likely to pick the
gift for a friend, while those who had held the cold pad tended to
choose the gift for themselves. This finding suggests that physical
warmth does not just influence how we see other people, but can also
affect our own behaviour, making us act in a more generous or prosocial
way.

\begin{quote}
{\#yourturn}

What do you think: How are physical warmth and prosocial giving related?
\end{quote}

The researchers expected that physical warmth would lead to more
generous behavior because our early life experiences often connect
warmth with comfort, safety, and care from others. For example, being
held close by a caregiver usually involves both physical warmth and
feelings of trust and affection. Over time, these experiences create an
unconscious link between physical warmth and social warmth.

In the study, participants who were in the warm condition were more
likely to make the generous, prosocial choice of giving a gift to a
friend, rather than keeping it for themselves. Together, these findings
suggest that physical warmth can unconsciously activate ideas of social
closeness and trust. This supports the broader idea that common
expressions like calling someone ``cold'' or ``warm'' are not just
figures of speech but may reflect real psychological processes.

From this we could conclude that physical warmth can lead to perceiving
others as ``warmer'' people and it also makes us ``warmer'' and more
generous.

\section{The Aftermath}\label{the-aftermath-8}

Given the striking and intuitive appeal of the original findings, they
received significant media attention and were cited widely. However, as
concerns about replicability in social psychology grew, so did scrutiny
of the hot coffee study. Scientists emphasized that the hypothesis that
hot coffee makes you generous is worth testing again, as the original
had several methodological flaws. A major limitation of the original
research was the small sample size. The two studies included only 41 and
53 participants, respectively. Small samples increase the risk that
results reflect random variation rather than a genuine effect, which
reduces the reliability of the conclusions. In other words, findings
based on so few participants may not hold up when tested with larger
groups.

A second issue was that the participants were not representative of the
wider population. They were all undergraduate students from one
university in New York State. College students often share similar age,
education level, and cultural background, which means the findings might
not apply to older adults, children, or people from other places and
backgrounds.

\begin{quote}
\phantomsection\label{def-representativity}{\#definition} Definition of
representativity

The extent to which a study sample reflects a well-defined target
population, such that the estimates or the interpretation of results can
be generalised to that population
(\citeproc{ref-rudolph_defining_2023}{Rudolph et al. 2023}).
\end{quote}

Replication studies attempted to address these limitations by recruiting
larger samples, with more than three times the original number of
participants, and by selecting more diverse populations. These
methodological improvements provided stronger statistical power and
greater external validity, allowing researchers to test whether the
effect was robust beyond the narrow conditions of the original
experiments.

Multiple high-powered replication attempts have since failed to
reproduce the original effects. In 2014, Lynott and colleagues conducted
a multi-lab replication of the first Williams and Bargh experiment. A
multi-lab replication is when several independent research groups carry
out the same study using a common protocol. This approach reduces the
likelihood that the outcome is due to local conditions or the influence
of a single research team. Across a larger and more diverse sample,
Lynott et al. (\citeproc{ref-lynott_replication_2014}{2014}) found no
evidence that holding a warm object influenced social judgments.

In 2018, Chabris et al. (\citeproc{ref-chabris_no_2019}{2019}) attempted
to replicate the findings of Williams and Bargh
(\citeproc{ref-williams_experiencing_2008}{2008}) using more rigorous
methodology. Their studies addressed several shortcomings of the
original. The first studies used very small samples (41 and 53 people),
which makes results unstable and prone to influences of chance. Chabris
et al.~tested much larger groups, giving their findings more statistical
power. The original participants were all college students, limiting
generalisability; the replication recruited a more diverse public
sample. The original studies also took place in a lab with experimenters
aware of conditions, raising concerns about artificiality and bias.
Chabris et al.~tested participants in a natural field setting and used
double-blind procedures. Despite these improvements, and contrary to the
original claims, they found no evidence that holding a hot or cold
object influenced participants' judgments or generosity. In other words,
the replication showed no evidence that physical warmth affected
behaviour or perceptions (\citeproc{ref-chabris_no_2019}{Chabris et al.
2019}).

\begin{quote}
{\#yourturn}

What could be the cause for a differing result?
\end{quote}

The replication researchers, however, do not conclude that hot coffee
does not make people generous. Instead, because they found null effects,
they concluded that there was no evidence for such an effect. Using a
different statistical approach, they found that the evidence actually
favored the interpretation that there was no effect, and not the
interpretation that they might have missed detecting the original
effect.

Some later research has suggested that warmth effects might still exist,
but only under specific conditions. A study by Citron and Goldberg
(\citeproc{ref-citron_social_2014}{2014}) found that physical warmth
increased perceptions of interpersonal kindness only in neutral social
contexts. When participants read about someone behaving negatively, the
warmth manipulation had no effect. This suggests that the influence of
physical warmth on social judgment is not universal, but shaped by the
surrounding social context.

\section{Conclusion}\label{conclusion-8}

The Hot Coffee study (\citeproc{ref-williams_experiencing_2008}{Williams
and Bargh 2008}) sparked fascination with the idea that physical
sensations shape social judgments. However, over a decade of follow-up
research has largely failed to replicate these findings reliably. While
the metaphor of warmth remains powerful in language and intuition, its
psychological effects appear to be fragile, context-sensitive, and not
easily reproduced under stricter experimental controls.

This case illustrates an important lesson in psychological science: even
intuitively satisfying findings must be rigorously tested, replicated,
and interpreted within a broader theoretical and methodological context.
The story of this study also reflects a broader shift in psychology:
moving away from surprising, single-study findings toward replication,
cumulative evidence, and methodological transparency.

\begin{quote}
{\#yourturn}

Which study seems more convincing to you? Why?
\end{quote}

More broadly, this debate encourages us to reflect on the role of the
environment in shaping human behaviour. Findings like the Hot Coffee
study suggest that seemingly minor physical cues might influence
judgments and actions, but the difficulty in replicating these effects
shows that such influences are neither simple nor uniform. Environmental
factors may interact with individual differences, situational contexts,
and cultural expectations in ways that make their effects less
predictable than early studies implied. Our thoughts and actions could
be shaped in subtle ways by the contexts we are in. The challenge for
psychology is to determine which of these effects are robust,
meaningful, and practically relevant, and which are not.

\bookmarksetup{startatroot}

\chapter*{\texorpdfstring{{Summary}}{Summary}}\label{summary}
\addcontentsline{toc}{chapter}{{Summary}}

\markboth{{Summary}}{{Summary}}

Add here

\section*{Take-Aways}\label{take-aways}
\addcontentsline{toc}{section}{Take-Aways}

\markright{Take-Aways}

Add here

\section*{Thanks}\label{thanks}
\addcontentsline{toc}{section}{Thanks}

\markright{Thanks}

This book was made possible by the many helping hands and critical
thoughts of the student authors involved in writing the individual
chapters. In addition, Melissa Engelbarth's support with selecting and
translating the chapters to include was invaluable.

\bookmarksetup{startatroot}

\chapter*{References}\label{references}
\addcontentsline{toc}{chapter}{References}

\markboth{References}{References}

\phantomsection\label{refs}
\begin{CSLReferences}{1}{0}
\bibitem[\citeproctext]{ref-Allport_1954}
Allport, Gordon W. 1954. {``The Nature of Prejudice,''} The nature of
prejudice, xviii, 537.

\bibitem[\citeproctext]{ref-anderson_temperature_nodate}
Anderson, C. A., and K. B. Anderson. 1998. {``Temperature and
Aggression: {Paradox}, Controversy, and a ({Fairly}) Clear Picture.''}
In \emph{Human Aggression: {Theories}, Research and Implications for
Policy}, edited by R. Geen and E. Donnerstein, 247--98. Academic Press.

\bibitem[\citeproctext]{ref-Anderson-Hanley_et_al_2011}
Anderson-Hanley, C., P. Arciero, and A. Snyder. 2011. {``Social
Facilitation in Virtual Reality-Enhanced Exercise: Competitiveness
Moderates Exercise Effort of Older Adults.''} \emph{Clinical
Interventions in Aging} 6: 275--80.
\url{https://doi.org/10.2147/CIA.S25337}.

\bibitem[\citeproctext]{ref-APA-dict_2018}
{``{APA Dictionary} of {Psychology}.''} 2018.
https://dictionary.apa.org/.
\url{https://dictionary.apa.org/association}.

\bibitem[\citeproctext]{ref-Aronson-et-al_2005}
Aronson, E., T. D. Wilson, and R. M. Akert. 2005. {``Social
Psychology.''}

\bibitem[\citeproctext]{ref-baker_negro-white_1934}
Baker, Paul E. 1934. {``Negro-{White} {Adjustment} in {America}.''}
\emph{The Journal of Negro Education} 3 (2): 194.
\url{https://doi.org/10.2307/2292313}.

\bibitem[\citeproctext]{ref-Baron_1986}
Baron, R. S. 1986. {``Distraction-Conflict Theory: Progress and
Problems.''} In \emph{Advances in Experimental Social Psychology},
19:1--40. Academic Press.
\url{https://doi.org/10.1016/S0065-2601(08)60211-7}.

\bibitem[\citeproctext]{ref-baumeister1998}
Baumeister, Roy F., Ellen Bratslavsky, Mark Muraven, and Dianne M. Tice.
1998. {``Ego Depletion: Is the Active Self a Limited Resource?''}
\emph{Journal of Personality and Social Psychology} 74 (5): 1252--65.
\url{https://doi.org/10.1037/0022-3514.74.5.1252}.

\bibitem[\citeproctext]{ref-bluxe1zquez2017}
Blázquez, Desirée, Juan Botella, and Manuel Suero. 2017. {``The Debate
on the Ego-Depletion Effect: Evidence from Meta-Analysis with the
p-Uniform Method.''} \emph{Frontiers in Psychology} 8 (February).
\url{https://doi.org/10.3389/fpsyg.2017.00197}.

\bibitem[\citeproctext]{ref-Bond-Titus_1983}
Bond, C. F., and L. J. Titus. 1983. {``Social Facilitation: A
Meta-Analysis of 241 Studies.''} \emph{Psychological Bulletin} 94 (2):
265--92. \url{https://doi.org/10.1037/0033-2909.94.2.265}.

\bibitem[\citeproctext]{ref-brameld_minority_1946}
Brameld, Theodore. 1946. \emph{Minority {Problems} in the {Public}
{Schools}: {A} {Study} of {Administrative} {Policies} and {Practices} in
{Seven} {School} {Systems}}. Harper \& Brothers.

\bibitem[\citeproctext]{ref-brandt_replication_2014}
Brandt, Mark J., Hans IJzerman, Ap Dijksterhuis, Frank J. Farach, Jason
Geller, Roger Giner-Sorolla, James A. Grange, Marco Perugini, Jeffrey R.
Spies, and Anna Van 'T Veer. 2014. {``The {Replication} {Recipe}: {What}
Makes for a Convincing Replication?''} \emph{Journal of Experimental
Social Psychology} 50 (January): 217--24.
\url{https://doi.org/10.1016/j.jesp.2013.10.005}.

\bibitem[\citeproctext]{ref-brophy_luxury_1945}
Brophy, Ira N. 1945. {``The {Luxury} of {Anti}-{Negro} {Prejudice}.''}
\emph{Public Opinion Quarterly} 9 (4): 456--66.
\url{https://doi.org/10.1086/265762}.

\bibitem[\citeproctext]{ref-Brophy_Good_1970}
Brophy, J. E., and T. L. Good. 1970. {``Teachers' Communication of
Differential Expectations for Children's Classroom Performance: Some
Behavioral Data.''} \emph{Journal of Educational Psychology} 61:
365--74. \url{https://doi.org/10.1037/h0029908}.

\bibitem[\citeproctext]{ref-bunker_how_2021}
Bunker, Cameron J., and Michael E. W. Varnum. 2021. {``How Strong Is the
Association Between Social Media Use and False Consensus?''}
\emph{Computers in Human Behavior} 125 (December): 106947.
\url{https://doi.org/10.1016/j.chb.2021.106947}.

\bibitem[\citeproctext]{ref-Cameron_Cook_2013}
Cameron, D. L., and B. G. Cook. 2013. {``General Education Teachers'
Goals and Expectations for Their Included Students with Mild and Severe
Disabilities.''} \emph{Education and Training in Autism and
Developmental Disabilities} 48: 18--30.
\url{http://www.jstor.org/stable/23879883}.

\bibitem[\citeproctext]{ref-carter2014}
Carter, Evan C., and Michael E. McCullough. 2014. {``Publication Bias
and the Limited Strength Model of Self-Control: Has the Evidence for Ego
Depletion Been Overestimated?''} \emph{Frontiers in Psychology} 5
(July). \url{https://doi.org/10.3389/fpsyg.2014.00823}.

\bibitem[\citeproctext]{ref-deCastro_1994}
Castro, J. M. de. 1994. {``Family and Friends Produce Greater Social
Facilitation of Food Intake Than Other Companions.''} \emph{Physiology
\& Behavior} 56 (3): 445--55.
\url{https://doi.org/10.1016/0031-9384(94)90286-0}.

\bibitem[\citeproctext]{ref-chabris_no_2019}
Chabris, Christopher F., Patrick R. Heck, Jaclyn Mandart, Daniel J.
Benjamin, and Daniel J. Simons. 2019. {``No {Evidence} {That}
{Experiencing} {Physical} {Warmth} {Promotes} {Interpersonal} {Warmth}:
{Two} {Failures} to {Replicate}.''} \emph{Social Psychology} 50 (2):
127--32. \url{https://doi.org/10.1027/1864-9335/a000361}.

\bibitem[\citeproctext]{ref-chambers2019registered}
Chambers, Chris. 2019. {``The Registered Reports Revolution Lessons in
Cultural Reform.''} \emph{Significance} 16 (4): 23--27.

\bibitem[\citeproctext]{ref-christensen2020open}
Christensen, Garret, Zenan Wang, Elizabeth Levy Paluck, Nicholas
Swanson, David Birke, Edward Miguel, and Rebecca Littman. 2020. {``Open
Science Practices Are on the Rise: The State of Social Science (3S)
Survey.''}

\bibitem[\citeproctext]{ref-citron_social_2014}
Citron, Francesca M. M., and Adele E. Goldberg. 2014. {``Social Context
Modulates the Effect of Physical Warmth on Perceived Interpersonal
Kindness: A Study of Embodied Metaphors.''} \emph{Language and
Cognition} 6 (1): 1--11. \url{https://doi.org/10.1017/langcog.2013.4}.

\bibitem[\citeproctext]{ref-coles_multi-lab_2022}
Coles, Nicholas A., David S. March, Fernando Marmolejo-Ramos, Jeff T.
Larsen, Nwadiogo C. Arinze, Izuchukwu L. G. Ndukaihe, Megan L. Willis,
et al. 2022. {``A Multi-Lab Test of the Facial Feedback Hypothesis by
the {Many} {Smiles} {Collaboration}.''} \emph{Nature Human Behaviour} 6
(12): 1731--42. \url{https://doi.org/10.1038/s41562-022-01458-9}.

\bibitem[\citeproctext]{ref-Cottrell_1972}
Cottrell, N. B. 1972. {``Social Facilitation.''} In \emph{Experimental
Social Psychology}, edited by C. G. McClintock. Holt, Rinehart \&
Winston.

\bibitem[\citeproctext]{ref-Cottrell-et-al_1968}
Cottrell, N. B., D. L. Wack, G. J. Sekerak, and R. H. Rittle. 1968.
{``Social Facilitation of Dominant Responses by the Presence of an
Audience and the Mere Presence of Others.''} \emph{Journal of
Personality and Social Psychology} 9 (3): 245--50.
\url{https://doi.org/10.1037/h0025902}.

\bibitem[\citeproctext]{ref-cumming_new_2014}
Cumming, Geoff. 2014. {``The New Statistics: Why and How.''}
\emph{Psychological Science} 25 (1): 7--29.
\url{https://doi.org/10.1177/0956797613504966}.

\bibitem[\citeproctext]{ref-dang2017a}
Dang, Junhua. 2017. {``An Updated Meta-Analysis of the Ego Depletion
Effect.''} \emph{Psychological Research} 82 (4): 645--51.
\url{https://doi.org/10.1007/s00426-017-0862-x}.

\bibitem[\citeproctext]{ref-darwin_expression_1872}
Darwin, Charles. 1872. \emph{The Expression of the Emotions in Man and
Animals}. The Expression of the Emotions in Man and Animals. London,
England: John Murray. \url{https://doi.org/10.1037/10001-000}.

\bibitem[\citeproctext]{ref-Dashiell_1930}
Dashiell, J. F. 1930. {``An Experimental Analysis of Some Group
Effects.''} \emph{The Journal of Abnormal and Social Psychology} 25 (2):
190--99. \url{https://doi.org/10.1037/h0075144}.

\bibitem[\citeproctext]{ref-De_Boer_Bosker_Van_der_Werf_2010}
De Boer, H., R. J. Bosker, and M. P. C. Van der Werf. 2010.
{``Sustainability of Teacher Expectation Bias Effects on Long-Term
Student Performance.''} \emph{Journal of Educational Psychology} 102:
168--79. \url{https://doi.org/10.1037/a0017289}.

\bibitem[\citeproctext]{ref-deutsch_interracial_1951}
Deutsch, Morton, and Mary Evans Collins. 1951. \emph{Interracial
Housing; a Psychological Evaluation of a Social Experiment}. Interracial
Housing; a Psychological Evaluation of a Social Experiment. Minneapolis,
MN, US: University of Minnesota Press.

\bibitem[\citeproctext]{ref-nathan_dewall_hot_2009}
DeWall, Nathan C., and Brad J. Bushman. 2009. {``Hot Under the Collar in
a Lukewarm Environment: {Words} Associated with Hot Temperature Increase
Aggressive Thoughts and Hostile Perceptions.''} \emph{Journal of
Experimental Social Psychology} 45 (4): 1045--47.
\url{https://doi.org/10.1016/j.jesp.2009.05.003}.

\bibitem[\citeproctext]{ref-Eden_1990}
Eden, D. 1990. \emph{Pygmalion in Management: Productivity as a
Self-Fulfilling Prophecy}.

\bibitem[\citeproctext]{ref-ekman_argument_1992}
Ekman, Paul. 1992. {``An Argument for Basic Emotions.''} \emph{Cognition
and Emotion} 6 (3-4): 169--200.
\url{https://doi.org/10.1080/02699939208411068}.

\bibitem[\citeproctext]{ref-Festinger_1954}
Festinger, L. 1954. {``A Theory of Social Comparison Processes.''}
\emph{Human Relations} 7 (2): 117--40.

\bibitem[\citeproctext]{ref-fiske_universal_2007}
Fiske, Susan T., Amy J. C. Cuddy, and Peter Glick. 2007. {``Universal
Dimensions of Social Cognition: Warmth and Competence.''} \emph{Trends
in Cognitive Sciences} 11 (2): 77--83.
\url{https://doi.org/10.1016/j.tics.2006.11.005}.

\bibitem[\citeproctext]{ref-friese2018}
Friese, Malte, David D. Loschelder, Karolin Gieseler, Julius
Frankenbach, and Michael Inzlicht. 2018. {``Is Ego Depletion Real? An
Analysis of Arguments.''} \emph{Personality and Social Psychology
Review} 23 (2): 107--31. \url{https://doi.org/10.1177/1088868318762183}.

\bibitem[\citeproctext]{ref-furnas_people_2024}
Furnas, Alexander C., and Timothy M. LaPira. 2024. {``The People Think
What {I} Think: {False} Consensus and Unelected Elite Misperception of
Public Opinion.''} \emph{American Journal of Political Science} 68 (3):
958--71. \url{https://doi.org/10.1111/ajps.12833}.

\bibitem[\citeproctext]{ref-Greenwald-et-al_1998}
Greenwald, Anthony G., Debbie E. McGhee, and Jordan L. K. Schwartz.
1998. {``Measuring Individual Differences in Implicit Cognition: {The}
Implicit Association Test.''} \emph{Journal of Personality and Social
Psychology} 74 (6): 1464--80.
\url{https://doi.org/10.1037/0022-3514.74.6.1464}.

\bibitem[\citeproctext]{ref-Greenwald-et-al_2003}
Greenwald, Anthony G., Brian A. Nosek, and Mahzarin R. Banaji. 2003.
{``Understanding and Using the {Implicit Association Test}: {I}. {An}
Improved Scoring Algorithm.''} \emph{Journal of Personality and Social
Psychology} 85 (2): 197--216.
\url{https://doi.org/10.1037/0022-3514.85.2.197}.

\bibitem[\citeproctext]{ref-Greenwald-et-al_2009}
Greenwald, Anthony G., T. Andrew Poehlman, Eric Luis Uhlmann, and
Mahzarin R. Banaji. 2009. {``Understanding and Using the {Implicit
Association Test}: {III}. {Meta}-Analysis of Predictive Validity.''}
\emph{Journal of Personality and Social Psychology} 97 (1): 17--41.
\url{https://doi.org/10.1037/a0015575}.

\bibitem[\citeproctext]{ref-hagger2010}
Hagger, Martin S., Chantelle Wood, Chris Stiff, and Nikos L. D.
Chatzisarantis. 2010. {``Ego Depletion and the Strength Model of
Self-Control: A Meta-Analysis.''} \emph{Psychological Bulletin} 136 (4):
495--525. \url{https://doi.org/10.1037/a0019486}.

\bibitem[\citeproctext]{ref-Halfmann-et-al_2020}
Halfmann, E., J. Bredehöft, and J. A. Häusser. 2020. {``Replicating
Roaches: A Preregistered Direct Replication of Zajonc, Heingartner, and
Herman's (1969) Social-Facilitation Study.''} \emph{Psychological
Science} 31 (3): 332--37.
\url{https://doi.org/10.1177/0956797620902101}.

\bibitem[\citeproctext]{ref-Hull_1943}
Hull, C. L. 1943. \emph{Principles of Behavior}. Appleton-Century
Crofts.

\bibitem[\citeproctext]{ref-inzlicht2015}
Inzlicht, Michael, Will Gervais, and Elliot Berkman. 2015.
{``Bias-Correction Techniques Alone Cannot Determine Whether Ego
Depletion Is Different from Zero: Commentary on Carter, Kofler, Forster,
\& Mccullough, 2015.''} \emph{SSRN Electronic Journal}.
\url{https://doi.org/10.2139/ssrn.2659409}.

\bibitem[\citeproctext]{ref-Izuma-et-al_2010}
Izuma, K., D. N. Saito, and N. Sadato. 2010. {``Processing of the
Incentive for Social Approval in the Ventral Striatum During Charitable
Donation.''} \emph{Journal of Cognitive Neuroscience} 22 (4): 621--31.
\url{https://doi.org/10.1162/jocn.2009.21228}.

\bibitem[\citeproctext]{ref-james_what_1884}
James, William. 1884. {``What Is an Emotion?''} \emph{Mind} os-IX (34):
188--205. \url{https://doi.org/10.1093/mind/os-IX.34.188}.

\bibitem[\citeproctext]{ref-Jenner_1990}
Jenner, H. 1990. {``The Pygmalion Effect: The Importance of
Expectancies.''} \emph{Alcoholism Treatment Quarterly} 7 (2): 127--33.

\bibitem[\citeproctext]{ref-Jhangiani-Tarry_2022}
Jhangiani, Dr Rajiv, and Dr Hammond Tarry. 2022. {``11.1 {Social
Categorization} and {Stereotyping}.''} \emph{Principles of Social
Psychology - 1st International H5P Edition}, January.

\bibitem[\citeproctext]{ref-Jussim_Harber_2005}
Jussim, L., and K. D. Harber. 2005. {``Teacher Expectations and
Self-Fulfilling Prophecies: Knowns and Unknowns, Resolved and Unresolved
Controversies.''} \emph{Personality and Social Psychology Review} 9:
131--55. \url{https://doi.org/10.1207/s15327957pspr0902_3}.

\bibitem[\citeproctext]{ref-kephart_racial_1957}
Kephart, William M. 1957. \emph{Racial {Factors} and {Urban} {Law}
{Enforcement}}. University of Pennsylvania Press.
\url{https://www.jstor.org/stable/j.ctv4w3vv0}.

\bibitem[\citeproctext]{ref-kidwell2016badges}
Kidwell, Mallory C, Ljiljana B Lazarević, Erica Baranski, Tom E
Hardwicke, Sarah Piechowski, Lina-Sophia Falkenberg, Curtis Kennett, et
al. 2016. {``Badges to Acknowledge Open Practices: A Simple, Low-Cost,
Effective Method for Increasing Transparency.''} \emph{PLoS Biology} 14
(5): e1002456.

\bibitem[\citeproctext]{ref-kuhn1962structure}
Kuhn, Thomas. 1962. {``The Structure of Scientific Revolutions.''}
\emph{International Encyclopedia of Unified Science} 2 (2).

\bibitem[\citeproctext]{ref-Learman_et_al_1990}
Learman, L. A., J. Avorn, D. E. Everitt, and R. Rosenthal. 1990.
{``Pygmalion in the Nursing Home: The Effects of Caregiver Expectations
on Patient Outcomes.''} \emph{Journal of the American Geriatrics
Society} 38 (7): 797--803.

\bibitem[\citeproctext]{ref-lett_techniques_1945}
Lett, Harold A. 1945. \emph{Techniques for {Achieving} {Interracial}
{Cooperation}}. \url{http://archive.org/details/Lett070}.

\bibitem[\citeproctext]{ref-lowe_has_2025}
Lowe, Matt. 2025. {``Has {Intergroup} {Contact} {Delivered}?''}
\emph{Annual Review of Economics} 17 (Volume 17, 2025): 321--44.
\url{https://doi.org/10.1146/annurev-economics-081324-091109}.

\bibitem[\citeproctext]{ref-luzsa_false_2021}
Luzsa, Robert, and Susanne Mayr. 2021. {``False Consensus in the Echo
Chamber: {Exposure} to Favorably Biased Social Media News Feeds Leads to
Increased Perception of Public Support for Own Opinions.''}
\emph{Cyberpsychology: Journal of Psychosocial Research on Cyberspace}
15 (1). \url{https://doi.org/10.5817/CP2021-1-3}.

\bibitem[\citeproctext]{ref-lynott_replication_2014}
Lynott, Dermot, Katherine S. Corker, Jessica Wortman, Louise Connell, M.
Brent Donnellan, Richard E. Lucas, and Kerry O'Brien. 2014.
{``Replication of {`{Experiencing} {Physical} {Warmth} {Promotes}
{Interpersonal} {Warmth}'} By.''} \emph{Social Psychology} 45 (3):
216--22. \url{https://doi.org/10.1027/1864-9335/a000187}.

\bibitem[\citeproctext]{ref-Madon_et_al_1997}
Madon, S., L. Jussim, and J. Eccles. 1997. {``In Search of the Powerful
Self-Fulfilling Prophecy.''} \emph{Journal of Personality and Social
Psychology} 72: 791--809.
\url{https://doi.org/10.1037/0022-3514.72.4.791}.

\bibitem[\citeproctext]{ref-mccarthy_close_2014}
McCarthy, Randy J. 2014. {``Close Replication Attempts of the Heat
Priming-Hostile Perception Effect.''} \emph{Journal of Experimental
Social Psychology} 54 (September): 165--69.
\url{https://doi.org/10.1016/j.jesp.2014.04.014}.

\bibitem[\citeproctext]{ref-McHugh_et_al_2025}
McHugh, C., S. M. Griffin, E. L. Kinsella, M. Quayle, B. Strunz, and T.
Muldoon Orla. 2025. {``A Replication of Triplett's {`Social Facilitation
Experiment'}.''} \emph{Scientific Reports} 15.
\url{https://doi.org/10.1038/s41598-025-25608-x}.

\bibitem[\citeproctext]{ref-mousa_building_2020}
Mousa, Salma. 2020. {``Building Social Cohesion Between {Christians} and
{Muslims} Through Soccer in Post-{ISIS} {Iraq}.''} \emph{Science} 369
(6505): 866--70. \url{https://doi.org/10.1126/science.abb3153}.

\bibitem[\citeproctext]{ref-mullen_false_1985}
Mullen, Brian, Jennifer L Atkins, Debbie S Champion, Cecelia Edwards,
Dana Hardy, John E Story, and Mary Vanderklok. 1985. {``The False
Consensus Effect: {A} Meta-Analysis of 115 Hypothesis Tests.''}
\emph{Journal of Experimental Social Psychology} 21 (3): 262--83.
\url{https://doi.org/10.1016/0022-1031(85)90020-4}.

\bibitem[\citeproctext]{ref-Muller_Butera_2007}
Muller, D., and F. Butera. 2007. {``The Focusing Effect of
Self-Evaluation Threat in Coaction and Social Comparison.''}
\emph{Journal of Personality and Social Psychology} 93 (2): 194--211.
\url{https://doi.org/10.1037/0022-3514.93.2.194}.

\bibitem[\citeproctext]{ref-munafo2017manifesto}
Munafò, Marcus R, Brian A Nosek, Dorothy VM Bishop, Katherine S Button,
Christopher D Chambers, Nathalie Percie du Sert, Uri Simonsohn, Eric-Jan
Wagenmakers, Jennifer J Ware, and John Ioannidis. 2017. {``A Manifesto
for Reproducible Science.''} \emph{Nature Human Behaviour} 1 (1): 1--9.

\bibitem[\citeproctext]{ref-noah_when_2018}
Noah, Tom, Yaacov Schul, and Ruth Mayo. 2018. {``When Both the Original
Study and Its Failed Replication Are Correct: {Feeling} Observed
Eliminates the Facial-Feedback Effect.''} \emph{Journal of Personality
and Social Psychology} 114 (5): 657--64.
\url{https://doi.org/10.1037/pspa0000121}.

\bibitem[\citeproctext]{ref-Nosek-Smyth_2007}
Nosek, Brian A., and Frederick L. Smyth. 2007. {``A
{Multitrait}-{Multimethod Validation} of the {Implicit Association
Test}.''} \emph{Experimental Psychology} 54 (1): 14--29.
\url{https://doi.org/10.1027/1618-3169.54.1.14}.

\bibitem[\citeproctext]{ref-Paluck-et-al_2019}
Paluck, Elizabeth Levy, Seth A. Green, and Donald P. Green. 2019. {``The
Contact Hypothesis Re-Evaluated.''} \emph{Behavioural Public Policy} 3
(2): 129--58. \url{https://doi.org/10.1017/bpp.2018.25}.

\bibitem[\citeproctext]{ref-paluck_prejudice_2021}
Paluck, Elizabeth Levy, Roni Porat, Chelsey S. Clark, and Donald P.
Green. 2021. {``Prejudice {Reduction}: {Progress} and {Challenges}.''}
\emph{Annual Review of Psychology} 72 (Volume 72, 2021): 533--60.
\url{https://doi.org/10.1146/annurev-psych-071620-030619}.

\bibitem[\citeproctext]{ref-paolini_negativity_2024}
Paolini, Stefania, Meghann Gibbs, Brett Sales, Danielle Anderson, and
Kylie McIntyre. 2024. {``Negativity Bias in Intergroup Contact:
{Meta}-Analytical Evidence That Bad Is Stronger Than Good, Especially
When People Have the Opportunity and Motivation to Opt Out of
Contact.''} \emph{Psychological Bulletin} 150 (8): 921--64.
\url{https://doi.org/10.1037/bul0000439}.

\bibitem[\citeproctext]{ref-Pettigrew-Tropp_2006}
Pettigrew, Thomas F., and Linda R. Tropp. 2006. {``A Meta-Analytic Test
of Intergroup Contact Theory.''} \emph{Journal of Personality and Social
Psychology} 90 (5): 751--83.
\url{https://doi.org/10.1037/0022-3514.90.5.751}.

\bibitem[\citeproctext]{ref-popper1959logic}
Popper, Karl R. 1959. \emph{The Logic of Scientific Discovery}.
Hutchinson \& Co.

\bibitem[\citeproctext]{ref-Raudenbush_1984}
Raudenbush, S. W. 1984. {``Magnitude of Teacher Expectancy Effects on
Pupil IQ as a Function of the Credibility of Expectancy Induction: A
Synthesis of Findings from 18 Experiments.''} \emph{Journal of
Educational Psychology} 76: 85--97.
\url{https://doi.org/10.1037/0022-0663.76.1.85}.

\bibitem[\citeproctext]{ref-Rosenbloom-et-al_2007}
Rosenbloom, T., A. Shahar, A. Perlman, D. Estreich, and E. Kirzner.
2007. {``Success on a Practical Driver's License Test with and Without
the Presence of Another Testee.''} \emph{Accident Analysis \&
Prevention} 39 (6): 1296--1301.
\url{https://doi.org/10.1016/j.aap.2007.03.015}.

\bibitem[\citeproctext]{ref-Rosenthal_Jacobson_1968}
Rosenthal, R., and L. Jacobson. 1968. {``Pygmalion in the Classroom.''}
\emph{The Urban Review} 3 (1): 16--20.
\url{https://doi.org/10.1007/bf02322211}.

\bibitem[\citeproctext]{ref-Rosenthal_Rubin_1978}
Rosenthal, R., and D. B. Rubin. 1978. {``Interpersonal Expectancy
Effects: The First 345 Studies.''} \emph{Behavioral and Brain Sciences}
1: 377--86. \url{https://doi.org/10.1017/S0140525X00075506}.

\bibitem[\citeproctext]{ref-ross_false_1977}
Ross, Lee, David Greene, and Pamela House. 1977. {``The {`False
Consensus Effect'}: {An} Egocentric Bias in Social Perception and
Attribution Processes.''} \emph{Journal of Experimental Social
Psychology} 13 (3): 279--301.
\url{https://doi.org/10.1016/0022-1031(77)90049-X}.

\bibitem[\citeproctext]{ref-Rubie-Davies_2007}
Rubie-Davies, C. M. 2007. {``Classroom Interactions: Exploring the
Practices of High- and Low-Expectation Teachers.''} \emph{British
Journal of Educational Psychology} 77: 289--306.
\url{https://doi.org/10.1348/000709906X101601}.

\bibitem[\citeproctext]{ref-rubie-davies_teacher_2015}
Rubie-Davies, Christine M., Elizabeth R. Peterson, Chris G. Sibley, and
Robert Rosenthal. 2015. {``A Teacher Expectation Intervention:
{Modelling} the Practices of High Expectation Teachers.''}
\emph{Contemporary Educational Psychology}, Examining
{Innovations}---{Navigating} the {Dynamic} {Complexities} of
{School}-{Based} {Intervention} {Research}, 40 (January): 72--85.
\url{https://doi.org/10.1016/j.cedpsych.2014.03.003}.

\bibitem[\citeproctext]{ref-rudolph_defining_2023}
Rudolph, Jacqueline E., Yongqi Zhong, Priya Duggal, Shruti H. Mehta, and
Bryan Lau. 2023. {``Defining Representativeness of Study Samples in
Medical and Population Health Research.''} \emph{BMJ Medicine} 2 (1).
\url{https://doi.org/10.1136/bmjmed-2022-000399}.

\bibitem[\citeproctext]{ref-Sanders-et-al_1978}
Sanders, G. S., R. S. Baron, and D. L. Moore. 1978. {``Distraction and
Social Comparison as Mediators of Social Facilitation Effects.''}
\emph{Journal of Experimental Social Psychology} 14 (3): 291--303.
\url{https://doi.org/10.1016/0022-1031(78)90017-3}.

\bibitem[\citeproctext]{ref-Schimmack_2021}
Schimmack, Ulrich. 2021. {``The {Implicit Association Test}: {A Method}
in {Search} of a {Construct}.''} \emph{Perspectives on Psychological
Science} 16 (2): 396--414.
\url{https://doi.org/10.1177/1745691619863798}.

\bibitem[\citeproctext]{ref-Seitchik-et-al_2017}
Seitchik, A. E., A. J. Brown, and S. G. Harkins. 2017. {``Social
Facilitation: Using the Molecular to Inform the Molar.''} In \emph{The
Oxford Handbook of Social Influence}, 183--203. Oxford University Press.

\bibitem[\citeproctext]{ref-Snow_1969}
Snow, R. E. 1969. {``Unfinished Pygmalion {[}Review of the Book
Pygmalion in the Classroom: Teacher Expectation and Pupils' Intellectual
Development, by r. Rosenthal \& l. Jacobson{]}.''} \emph{Contemporary
Psychology} 14: 197--200. \url{https://doi.org/10.1037/0010293}.

\bibitem[\citeproctext]{ref-Sommer-et-al_1992}
Sommer, R., M. Wynes, and G. Brinkley. 1992. {``Social Facilitation
Effects in Shopping Behavior.''} \emph{Environment and Behavior} 24 (3):
285--97. \url{https://doi.org/10.1177/0013916592243001}.

\bibitem[\citeproctext]{ref-Spence_1956}
Spence, K. W. 1956. \emph{Behavior Theory and Conditioning}. Yale
University Press. \url{https://doi.org/10.1037/10029-000}.

\bibitem[\citeproctext]{ref-srull_role_1979}
Srull, Thomas K., and Robert S. Wyer. 1979. {``The Role of Category
Accessibility in the Interpretation of Information about Persons: {Some}
Determinants and Implications.''} \emph{Journal of Personality and
Social Psychology} 37 (10): 1660--72.
\url{https://doi.org/10.1037/0022-3514.37.10.1660}.

\bibitem[\citeproctext]{ref-steiner_false_2025}
Steiner, Nils D., Claudia Landwehr, and Philipp Harms. 2025. {``False
Consensus Beliefs and Populist Attitudes.''} \emph{Political Psychology}
00: 1--22. \url{https://doi.org/10.1111/pops.70026}.

\bibitem[\citeproctext]{ref-strack_reflection_2016}
Strack, Fritz. 2016. {``Reflection on the {Smiling} {Registered}
{Replication} {Report}.''} \emph{Perspectives on Psychological Science}
11 (6): 929--30. \url{https://doi.org/10.1177/1745691616674460}.

\bibitem[\citeproctext]{ref-strack_inhibiting_1988}
Strack, Fritz, Leonard L. Martin, and Sabine Stepper. 1988.
{``Inhibiting and Facilitating Conditions of the Human Smile: {A}
Nonobtrusive Test of the Facial Feedback Hypothesis.''} \emph{Journal of
Personality and Social Psychology} 54 (5): 768--77.
\url{https://doi.org/10.1037/0022-3514.54.5.768}.

\bibitem[\citeproctext]{ref-Strube_2005}
Strube, M. J. 2005. {``What Did Triplett Really Find? A Contemporary
Analysis of the First Experiment in Social Psychology.''} \emph{The
American Journal of Psychology} 118 (2): 271--86.

\bibitem[\citeproctext]{ref-Thorndike_1968}
Thorndike, R. L. 1968. {``{[}Review of the Book Pygmalion in the
Classroom, by r. Rosenthal \& l. Jacobson{]}.''} \emph{American
Educational Research Journal} 5: 708--11.
\url{https://doi.org/10.3102/00028312005004708}.

\bibitem[\citeproctext]{ref-Triplett_1898}
Triplett, N. 1898. {``The Dynamogenic Factors in Pacemaking and
Competition.''} \emph{The American Journal of Psychology} 9 (4):
507--33. \url{https://doi.org/10.2307/1412188}.

\bibitem[\citeproctext]{ref-tulving_priming_1982}
Tulving, Endel, Daniel L. Schacter, and Heather A. Stark. 1982.
{``Priming Effects in Word-Fragment Completion Are Independent of
Recognition Memory.''} \emph{Journal of Experimental Psychology:
Learning, Memory, and Cognition} 8 (4): 336--42.
\url{https://doi.org/10.1037/0278-7393.8.4.336}.

\bibitem[\citeproctext]{ref-vazire2018}
Vazire, Simine. 2018. {``Implications of the Credibility Revolution for
Productivity, Creativity, and Progress.''} \emph{Perspectives on
Psychological Science} 13 (4): 411--17.
\url{https://doi.org/10.1177/1745691617751884}.

\bibitem[\citeproctext]{ref-vazire2022credibility}
Vazire, Simine, Sarah R Schiavone, and Julia G Bottesini. 2022.
{``Credibility Beyond Replicability: Improving the Four Validities in
Psychological Science.''} \emph{Current Directions in Psychological
Science} 31 (2): 162--68.

\bibitem[\citeproctext]{ref-vohs2021}
Vohs, Kathleen D., Brandon J. Schmeichel, Sophie Lohmann, Quentin F.
Gronau, Anna J. Finley, Sarah E. Ainsworth, Jessica L. Alquist, et al.
2021. {``A Multisite Preregistered Paradigmatic Test of the
Ego-Depletion Effect.''} \emph{Psychological Science} 32 (10): 1566--81.
\url{https://doi.org/10.1177/0956797621989733}.

\bibitem[\citeproctext]{ref-wagenmakers_registered_2016}
Wagenmakers, E.-J., T. Beek, L. Dijkhoff, Q. F. Gronau, A. Acosta, R. B.
Adams, D. N. Albohn, et al. 2016. {``Registered {Replication} {Report}:
{Strack}, {Martin}, \& {Stepper} (1988).''} \emph{Perspectives on
Psychological Science} 11 (6): 917--28.
\url{https://doi.org/10.1177/1745691616674458}.

\bibitem[\citeproctext]{ref-wasserstein_asa_2016}
Wasserstein, Ronald L., and Nicole A. Lazar. 2016. {``The {ASA}
{Statement} on p-{Values}: {Context}, {Process}, and {Purpose}.''}
\emph{The American Statistician} 70 (2): 129--33.
\url{https://doi.org/10.1080/00031305.2016.1154108}.

\bibitem[\citeproctext]{ref-weinschenk_democratic_2021}
Weinschenk, Aaron C., Costas Panagopoulos, and Sander van der Linden.
2021. {``Democratic {Norms}, {Social} {Projection}, and {False}
{Consensus} in the 2020 {U}.{S}. {Presidential} {Election}.''}
\emph{Journal of Political Marketing} 20 (3-4): 255--68.
\url{https://doi.org/10.1080/15377857.2021.1939568}.

\bibitem[\citeproctext]{ref-Weiss_Miller_1971}
Weiss, R. F., and F. G. Miller. 1971. {``The Drive Theory of Social
Facilitation.''} \emph{Psychological Review} 78 (1): 44--57.
\url{https://doi.org/10.1037/h0030386}.

\bibitem[\citeproctext]{ref-williams_experiencing_2008}
Williams, Lawrence E., and John A. Bargh. 2008. {``Experiencing
{Physical} {Warmth} {Promotes} {Interpersonal} {Warmth}.''}
\emph{Science} 322 (5901): 606--7.
\url{https://doi.org/10.1126/science.1162548}.

\bibitem[\citeproctext]{ref-Zajonc_1965}
Zajonc, R. B. 1965. {``Social Facilitation.''} \emph{Science} 149
(3681): 269--74. \url{https://doi.org/10.1126/science.149.3681.269}.

\end{CSLReferences}

\bookmarksetup{startatroot}

\chapter*{\texorpdfstring{{Glossary}}{Glossary}}\label{glossary}
\addcontentsline{toc}{chapter}{{Glossary}}

\markboth{{Glossary}}{{Glossary}}

\subsubsection*{\texorpdfstring{\hyperref[def-availabilityheuristic]{Availability
Heuristic}}{Availability Heuristic}}\label{availability-heuristic}
\addcontentsline{toc}{subsubsection}{\hyperref[def-availabilityheuristic]{Availability
Heuristic}}

A mental shortcut where people estimate the likelihood of an event based
on how easily examples come to mind, which can lead to overestimating
rare but memorable occurrences.

\subsubsection*{\texorpdfstring{\hyperref[def-attitude]{Attitude}}{Attitude}}\label{attitude}
\addcontentsline{toc}{subsubsection}{\hyperref[def-attitude]{Attitude}}

The cognition, affect and behavioral tendencies towards a certain
object.

\subsubsection*{\texorpdfstring{\hyperref[def-bias]{Bias}}{Bias}}\label{bias}
\addcontentsline{toc}{subsubsection}{\hyperref[def-bias]{Bias}}

A systematic distortion of perception or judgment.

\subsubsection*{\texorpdfstring{\hyperref[def-cherry-picking]{Cherry-Picking}}{Cherry-Picking}}\label{cherry-picking}
\addcontentsline{toc}{subsubsection}{\hyperref[def-cherry-picking]{Cherry-Picking}}

Reporting only the data, outcomes, or time frames that support one's
hypothesis while ignoring or dismissing those that do not. This makes
the story or articles simpler and might make them more publishable, but
provides a distorted view of the evidence.

\subsubsection*{\texorpdfstring{\hyperref[def-conceptualreplication]{Conceptual
Replication}}{Conceptual Replication}}\label{conceptual-replication}
\addcontentsline{toc}{subsubsection}{\hyperref[def-conceptualreplication]{Conceptual
Replication}}

A study that aims to recreate the gist of a prior study without using an
identical procedure. These studies often aim to explore boundary
conditions, the influence of specific variables, or aim to broaden and
extend a certain finding.

\subsubsection*{\texorpdfstring{Contact Interventions
(\hyperref[def-contactinterventions]{Chapter 12}
\hyperref[def-contactinterventions]{Chapter
20})}{Contact Interventions (Chapter 12 Chapter 20)}}\label{contact-interventions-chapter-12-chapter-20}
\addcontentsline{toc}{subsubsection}{Contact Interventions
(\hyperref[def-contactinterventions]{Chapter 12}
\hyperref[def-contactinterventions]{Chapter 20})}

Carefully tailored interventions that apply intergroup contact in
real-world settings to try and reduce prejudice among social groups.

\subsubsection*{\texorpdfstring{\hyperref[def-constructvalidity]{Construct
Validity}}{Construct Validity}}\label{construct-validity}
\addcontentsline{toc}{subsubsection}{\hyperref[def-constructvalidity]{Construct
Validity}}

The extent to which a test measures the theoretical construct or concept
it is intended to measure.

\subsubsection*{\texorpdfstring{\hyperref[def-discriminantvalidity]{Discriminant
Validity}}{Discriminant Validity}}\label{discriminant-validity}
\addcontentsline{toc}{subsubsection}{\hyperref[def-discriminantvalidity]{Discriminant
Validity}}

The extent to which a test is unrelated to measures designed to assess
theoretically distinct constructs.

\subsubsection*{\texorpdfstring{\hyperref[def-dynamogenesis]{Dynamogenesis}}{Dynamogenesis}}\label{dynamogenesis}
\addcontentsline{toc}{subsubsection}{\hyperref[def-dynamogenesis]{Dynamogenesis}}

An increase in the mental or motor activity of an already functioning
bodily system that accompanies any added sensory stimulation.

\subsubsection*{\texorpdfstring{\hyperref[def-effectsize]{Effect
Size}}{Effect Size}}\label{effect-size}
\addcontentsline{toc}{subsubsection}{\hyperref[def-effectsize]{Effect
Size}}

A quantitative measure of the magnitude of a phenomenon, used to assess
the practical significance of research findings.

\subsubsection*{\texorpdfstring{\hyperref[def-editorial]{Editorial}}{Editorial}}\label{editorial}
\addcontentsline{toc}{subsubsection}{\hyperref[def-editorial]{Editorial}}

An introductory article written by the editors of a special issue in an
academic journal. It outlines the purpose, scope, and significance of
the special issue, provides an overview of the included articles, and
often highlights key themes, trends, or gaps in the research field.

\subsubsection*{\texorpdfstring{\hyperref[def-egodepletion]{Ego
Depletion}}{Ego Depletion}}\label{ego-depletion-1}
\addcontentsline{toc}{subsubsection}{\hyperref[def-egodepletion]{Ego
Depletion}}

A concept that describes willpower as a limited resource that can be
used up (depleted).

\subsubsection*{\texorpdfstring{\hyperref[def-experiment]{Experiment}}{Experiment}}\label{experiment}
\addcontentsline{toc}{subsubsection}{\hyperref[def-experiment]{Experiment}}

A study where researchers deliberately manipulate one or more variables
and randomly assign participants to different conditions. Random
assignment helps ensure the groups are similar before the intervention,
so differences in outcomes are more likely to be caused by the
manipulation rather than by pre-existing differences.

\subsubsection*{\texorpdfstring{\hyperref[def-experiment]{Experiment}}{Experiment}}\label{experiment-1}
\addcontentsline{toc}{subsubsection}{\hyperref[def-experiment]{Experiment}}

A study where researchers deliberately manipulate one or more variables
and randomly assign participants to different conditions. Random
assignment helps ensure the groups are similar before the intervention,
so differences in outcomes are more likely to be caused by the
manipulation rather than by pre-existing differences.

\subsubsection*{\texorpdfstring{\hyperref[def-falseconsensus]{False
Consensus
Effect}}{False Consensus Effect}}\label{false-consensus-effect-1}
\addcontentsline{toc}{subsubsection}{\hyperref[def-falseconsensus]{False
Consensus Effect}}

A cognitive bias where individuals overestimate the extent to which
others share their beliefs, preferences, and behaviors.

\subsubsection*{\texorpdfstring{\hyperref[def-generalizeddrive]{Generalized
Drive}}{Generalized Drive}}\label{generalized-drive}
\addcontentsline{toc}{subsubsection}{\hyperref[def-generalizeddrive]{Generalized
Drive}}

The presence of others leads to an increase in generalized drive, thus
facilitating habitualized dominant responses.

\subsubsection*{\texorpdfstring{\hyperref[def-implicitassociationtest]{Implicit
Association
Test}}{Implicit Association Test}}\label{implicit-association-test}
\addcontentsline{toc}{subsubsection}{\hyperref[def-implicitassociationtest]{Implicit
Association Test}}

A reaction-time task that measures the strength of automatic
associations between concepts (e.g., flowers and positivity) by
comparing how quickly people classify paired categories. Faster
responses indicate stronger underlying associations.

\subsubsection*{\texorpdfstring{\hyperref[def-implicitattitude]{Implicit
Attitude}}{Implicit Attitude}}\label{implicit-attitude}
\addcontentsline{toc}{subsubsection}{\hyperref[def-implicitattitude]{Implicit
Attitude}}

An enduring mental disposition toward something that is not consciously
identified and of which a person may lack awareness.

\subsubsection*{\texorpdfstring{\hyperref[def-implicitsocialcognition]{Implicit
Social
Cognition}}{Implicit Social Cognition}}\label{implicit-social-cognition}
\addcontentsline{toc}{subsubsection}{\hyperref[def-implicitsocialcognition]{Implicit
Social Cognition}}

The automatic, unconscious mental processes that influence how we
perceive, evaluate, and interact with others.

\subsubsection*{\texorpdfstring{\hyperref[def-incongruentassociation]{Incongruent
Association}}{Incongruent Association}}\label{incongruent-association}
\addcontentsline{toc}{subsubsection}{\hyperref[def-incongruentassociation]{Incongruent
Association}}

A mental relationship between two objects or concepts characterized by
lack of harmony or misalignment.

\subsubsection*{\texorpdfstring{\hyperref[def-intergroupbias]{Intergroup
Bias}}{Intergroup Bias}}\label{intergroup-bias}
\addcontentsline{toc}{subsubsection}{\hyperref[def-intergroupbias]{Intergroup
Bias}}

Tendency to favour one's own social group (ingroup) over other groups
(outgroups), which often leads to negative attitudes or behaviours
toward outgroup members.

\subsubsection*{\texorpdfstring{\hyperref[def-intergroupbias]{Intergroup
Bias}}{Intergroup Bias}}\label{intergroup-bias-1}
\addcontentsline{toc}{subsubsection}{\hyperref[def-intergroupbias]{Intergroup
Bias}}

Tendency to favour one's own social group (ingroup) over other groups
(outgroups), which often leads to negative attitudes or behaviours
toward outgroup members.

\subsubsection*{\texorpdfstring{Meta-analysis
(\hyperref[def-meta-analysis]{Chapter 12}
\hyperref[def-meta-analysis]{Chapter
20})}{Meta-analysis (Chapter 12 Chapter 20)}}\label{meta-analysis-chapter-12-chapter-20}
\addcontentsline{toc}{subsubsection}{Meta-analysis
(\hyperref[def-meta-analysis]{Chapter 12}
\hyperref[def-meta-analysis]{Chapter 20})}

A statistical technique that combines the results of multiple
independent studies to estimate an overall effect. Meta-analyses can
reveal patterns across a large body of research, but the quality of
their conclusions depends on the quality and comparability of the
included studies.

\subsubsection*{\texorpdfstring{\hyperref[def-moderator]{Moderator}}{Moderator}}\label{moderator}
\addcontentsline{toc}{subsubsection}{\hyperref[def-moderator]{Moderator}}

A variable that modifies the relationship between independent and
dependent variables.

\subsubsection*{\texorpdfstring{\hyperref[def-multilabstudy]{Multi-Lab
Study}}{Multi-Lab Study}}\label{multi-lab-study}
\addcontentsline{toc}{subsubsection}{\hyperref[def-multilabstudy]{Multi-Lab
Study}}

A research project in which researchers working at several different
locations (laboratories) implement the same experimental design and then
analyse the data together.

\subsubsection*{\texorpdfstring{\hyperref[def-multimethodstudy]{Multimethod
Study}}{Multimethod Study}}\label{multimethod-study}
\addcontentsline{toc}{subsubsection}{\hyperref[def-multimethodstudy]{Multimethod
Study}}

Research that employs two or more distinct methods.

\subsubsection*{\texorpdfstring{\hyperref[def-observationalresearch]{Observational
Research}}{Observational Research}}\label{observational-research}
\addcontentsline{toc}{subsubsection}{\hyperref[def-observationalresearch]{Observational
Research}}

A study design where researchers measure variables as they naturally
occur, without manipulating them. Observational studies can reveal
associations between variables but cannot, on their own, establish that
one causes the other.

\subsubsection*{\texorpdfstring{\hyperref[def-pygmalioneffect]{Pygmalion
Effect}}{Pygmalion Effect}}\label{pygmalion-effect-1}
\addcontentsline{toc}{subsubsection}{\hyperref[def-pygmalioneffect]{Pygmalion
Effect}}

The phenomenon in which higher expectations from others lead to improved
performance.

\subsubsection*{\texorpdfstring{\hyperref[def-prejudice]{Prejudice}}{Prejudice}}\label{prejudice}
\addcontentsline{toc}{subsubsection}{\hyperref[def-prejudice]{Prejudice}}

A negative attitude toward a group and its members, often based on
stereotypes rather than direct experience.

\subsubsection*{\texorpdfstring{\hyperref[def-priming]{Priming}}{Priming}}\label{priming}
\addcontentsline{toc}{subsubsection}{\hyperref[def-priming]{Priming}}

A psychological phenomenon where exposure to one stimulus (e.g., a word,
image, or idea) influences how you respond to a later stimulus, often
without conscious awareness.

\subsubsection*{\texorpdfstring{\hyperref[def-representativity]{Representativity}}{Representativity}}\label{representativity}
\addcontentsline{toc}{subsubsection}{\hyperref[def-representativity]{Representativity}}

The extent to which a study sample reflects a well-defined target
population, such that the estimates or the interpretation of results can
be generalised to that population (Rudolph, J. E., 2023).

\subsubsection*{\texorpdfstring{\hyperref[def-socialcomparisontheory]{Social
Comparison
Theory}}{Social Comparison Theory}}\label{social-comparison-theory}
\addcontentsline{toc}{subsubsection}{\hyperref[def-socialcomparisontheory]{Social
Comparison Theory}}

According to the social comparison theory, people are motivated to
assess their own beliefs and skills by comparing them to external
images. These images can be comparisons to other people or a reference
to physical reality. Individuals have a tendency to view images
portrayed by others as accessible and realistic and subsequently make
comparisons between themselves, other people, and these idealized
images.

\subsubsection*{\texorpdfstring{\hyperref[def-socialdesirability]{Social
Desirability}}{Social Desirability}}\label{social-desirability}
\addcontentsline{toc}{subsubsection}{\hyperref[def-socialdesirability]{Social
Desirability}}

The tendency to want to be viewed positively by others, often by
aligning with socially approved behaviors and attitudes.

\subsubsection*{\texorpdfstring{\hyperref[def-sociallydesirableresponding]{Socially
Desirable
Responding}}{Socially Desirable Responding}}\label{socially-desirable-responding}
\addcontentsline{toc}{subsubsection}{\hyperref[def-sociallydesirableresponding]{Socially
Desirable Responding}}

The act of providing inauthentic responses to better present oneself
favorably according to current social norms.

\subsubsection*{\texorpdfstring{\hyperref[def-specialissue]{Special
Issue}}{Special Issue}}\label{special-issue}
\addcontentsline{toc}{subsubsection}{\hyperref[def-specialissue]{Special
Issue}}

A collection of articles on a specific topic, typically published
together in a single issue of an academic journal. Special issues are
often edited by guest editors and aim to provide a comprehensive
exploration of the chosen theme or field of study.

\subsubsection*{\texorpdfstring{\hyperref[def-treatmentandcontrolconditions]{Treatment
and Control
Conditions}}{Treatment and Control Conditions}}\label{treatment-and-control-conditions}
\addcontentsline{toc}{subsubsection}{\hyperref[def-treatmentandcontrolconditions]{Treatment
and Control Conditions}}

In experimental design, treatment condition refers to the participants
who are randomly chosen to undergo the intervention (e.g., to play in
the mixed soccer team). Control condition refers to the participants who
are subject to intervention-like treatment that lacks the critical
aspect of the intervention (e.g., those allocated to play in the
all-Christian soccer team). Here, the critical aspect is intergroup
contact within the team.

\subsubsection*{\texorpdfstring{\hyperref[def-treatmentandcontrolconditions]{Treatment
and Control
Conditions}}{Treatment and Control Conditions}}\label{treatment-and-control-conditions-1}
\addcontentsline{toc}{subsubsection}{\hyperref[def-treatmentandcontrolconditions]{Treatment
and Control Conditions}}

In experimental design, treatment condition refers to the participants
who are randomly chosen to undergo the intervention (e.g., to play in
the mixed soccer team). Control condition refers to the participants who
are subject to intervention-like treatment that lacks the critical
aspect of the intervention (e.g., those allocated to play in the
all-Christian soccer team). Here, the critical aspect is intergroup
contact within the team.

\subsubsection*{\texorpdfstring{\hyperref[def-zygomaticusmajormuscles]{Zygomaticus
Major
Muscles}}{Zygomaticus Major Muscles}}\label{zygomaticus-major-muscles}
\addcontentsline{toc}{subsubsection}{\hyperref[def-zygomaticusmajormuscles]{Zygomaticus
Major Muscles}}

These bilateral facial muscles, when activated, raise the corners of the
mouth in an upward and lateral direction, facilitating expressions such
as smiling.

\subsubsection*{\texorpdfstring{\hyperref[def-orbicularisoris]{Orbicularis
Oris
Muscles}}{Orbicularis Oris Muscles}}\label{orbicularis-oris-muscles}
\addcontentsline{toc}{subsubsection}{\hyperref[def-orbicularisoris]{Orbicularis
Oris Muscles}}

These are circular muscles around the mouth that close the lips and
produce puckering, as in kissing or whistling.


\backmatter

\end{document}
